% Kopfzeile beim Kapitelanfang:
\fancypagestyle{plain}{
%Kopfzeile links bzw. innen
\fancyhead[L]{\calligra\Large Vorlesung Nr. 6}
%Kopfzeile rechts bzw. außen
\fancyhead[R]{\calligra\Large 31.10.2013}
}
%Kopfzeile links bzw. innen
\fancyhead[L]{\calligra\Large Vorlesung Nr. 6}
%Kopfzeile rechts bzw. außen
\fancyhead[R]{\calligra\Large 31.10.2013}
% **************************************************
%
Abb. (Funktion) $f$ von $X$ nach $Y$ ($X, Y$ Mengen)\\
$f:X\to Y, x\mapsto f(x)$\\
\subsubsection*{Beispiel}
\enum{
\item $X = \{\text{Studenten an der UPB}\}, f:X\to \N, x\mapsto \text{Alle von }x$
\item $f:\R \to \R, x\to x^2$\\
$\Gamma_f:$ Normalparabel\\
$f(\R) = \{x\in \R: x≥0 \}$\\
\begin{tikzpicture}[domain=0:2, scale=0.5, prefix="plots/", smooth]
    \draw[very thin,color=gray] (-0.1,-0.1) grid (4.1,4.1);
    \draw[->] (-0.2,0) -- (4.2,0) node[right] {$x$};
    \draw[->] (0,-0.1) -- (0,4.2) node[right] {$y$};
    \draw[color=red] plot[id=v06g1] function{x*x} node[right] {};
\end{tikzpicture}
\item Lineare Funktionen\\
$f: \R \to \R, x \to a \cdot x + b$\\
$\Gamma_f:$ Gerade\\
$a:$ Steigung\\
$b$: Achsenabschnitt\\
%TODO Steigungsdreieck einzeichnen.
\begin{tikzpicture}[domain=0:4, scale=0.5, prefix="plots/", smooth]
    \draw[very thin,color=gray] (-0.1,-0.1) grid (4.1,4.1);
    \draw[->] (-0.2,0) -- (4.2,0) node[right] {$x$};
    \draw[->] (0,-0.1) -- (0,4.2) node[right] {$y$};
    \draw[color=red] plot[id=v06g2] function{ 0.7*x + 1} node[right] {};
\end{tikzpicture}
\item Polynomfunktionen
$p: \R \to \R, p(x) = a_nx^n + \cdots + a_1x + a_0$ mit $n\nN_0, a_0 ... a_n Koeffizienten$
\item Floor-Funktion (Gauß-Klammer)
$\lfloor . \rfloor: \R \to \R$\\
$\lfloor x \rfloor := größtes k\in \Z mit k ≤ x$\\
\begin{tikzpicture}[domain=0:4, scale=0.5, prefix="plots/", samples=250]
    \draw[very thin,color=gray] (-0.1,-0.1) grid (4.1,4.1);
    \draw[->] (-0.2,0) -- (4.2,0) node[right] {$x$};
    \draw[->] (0,-0.1) -- (0,4.2) node[right] {$y$};
    \draw[color=red] plot[id=v06g3] function{floor(x)} node[right] {};
\end{tikzpicture}
\item Identische Abbildung auf einer Menge $X$\\
$id_x: X\to X, x\to x$
\item $P:=$ Menge aller Sortierprogramme für endliche Listen\\
$L:= P\times \N \to \R$\\
$L(p, n) :=$ max. Laufzeit, die ein Programm $p$ zum sortieren einer Liste der Länge $n$ braucht.
}
Für die Charakterisierung einer Abbildung ist neben der Abbildungsvorschrift $f$ auch der Definitionsbereich $X$ wichtig.
\subsubsection*{Beispiel}
$f:\R \to \R, f(x) = |x|$
und $g:[-1, 1] \to \R, g(x) = |x|$ sind verschiedene Abbildungen. $g$ ist eine Restriktion von $f$.
\section{Definition}
Sei $f: X\to Y$ Abbildung\\
\enum {
\item Sei $A\subseteq X, f(A):= \{f(x) : x \in A\}$\\
$f(x) \subseteq Y:$ Wertebereich von $f$\\
\item Sei $B\subseteq Y: f^{-1}(B):=\{x\in X: f(x) \in B\}$\\
Urbild von $B$ unter $f$.\\
Besagt nicht, dass $f^{-1}$ Abbildung!\\
}
\subsubsection*{Beispiel}
$f:\Z \to \Z, x \mapsto x^2$\\
$f(\{-2, 5\}) = \{4, 25\}$\\
$f^{-1}(\{4, 25\}) = \{± 2, ±5\}$\\
$f^{-1}(\{3\}) = \emptyset$
\section*{Komposition von Abbildungen}
\section{Definition}
Seien $f:X\to Y$ und $g: Y\to Z$ Abbildungen\\
Die Komposition (Verknüpfung, Verkettung) von $f$ und $g$ ist die Abbildung\\
$g\circ f: X\to Z, x \to g(f(x))$\\
%TODO Schaubild Transitive Verkettung
($g$ nach $f$)\\
\subsubsection*{Beispiel}
$f, g: \R \to \R, f(x) = 2x+1 , g(x) = x^2$\\
$(g \circ f)(x) = (2x+1)^2 = 4x^2 + 4x + 1$\\
$(f \circ g)(x) = 2x^2 + 1$\\
$\Rarr (g \circ f) \neq (f \circ g)$\\
Aber Komposition ist Assoziativ
\section{Satz}
Seien $f:X\to Y, g:Y\to Z, h:Z\to W$ Abbildungen $\Rarr$ $h \circ (g \circ f) = (h \circ g) \circ f$\\
Beweis: $(h\circ (g\circ f))(x) = h((g\circ f)(x) = h((g(f(x)))) = (h \circ g)(f(x)) = ((h \circ g) \circ f)(x) \qed$
\section{Eigenschaften von Abbildungen}
\begin{description}
\item[injektiv] falls es zu jedem $y \in Y$ höchstens ein $x \in X$ gibt mit $f(x) = y$
\item[surjektiv] falls es zu jedem $y \in Y$ mindestens ein $x \in X$ gibt mit $f(x) = y$
\item[bijektiv] falls es zu jedem $y \in Y$ genau ein $x \in X$ gibt mit $f(x) = y$\\
$f(x)$ ist surjektiv und bijektiv
\end{description}
Sei $f:X\to Y$ bijektivm $\forall y\in Y \exists! x\in X: f(x) = y$\\
Wir können dann $g: Y\to X$ definieren durch $g(y) := x$ falls $y = f(x)$\\
damit $g(f(x)) = x\ \forall x\in X, f(g(y)) = y \forall y\in Y$\\
Das heißt: $g\circ f = id_x$, $f \circ g = id_y$
\section{Definition: Umkehrabbildung}
$g$ heißt die Umkehrabbildung (Umkehrfunktion) von $f$. \desc{Bezeichnung:}{$g= f^{-1}$}
Damit falls $f$ bijektiv: $y = f(x) \Leftrightarrow x = f^{-1}(y)$
\section{Satz}
Für $f: X\to Y$ sind äquivalent:
\enum {
\item $f$ ist bijektiv
\item $\exists g: Y\to X: g\circ f = id_x, f \circ g = id_y$
 in diesem Fall ist $g = f^{-1}$
}
\subsubsection*{Beweis}
\begin{description}
\item[1 $\Rarr$ 2] siehe oben
\item[2 $\Rarr$ 1] $f$ injektiv, denn: sei $f(x_1) = f(x_2) \Rarr x_1 = g(f(x_1)) = g(f(x_2)) = x_2$\\
$f$ surjektiv, denn: Sei $y \in Y \Rarr y = f(g(y))$\\
$g$ in (2) ist eindeutig, da $g(f(x)) = x$ und $f$ surjektiv$\qed$
\end{description}
Sei $f:X\to Y$ bijektiv\\
$X, Y \subseteq \R$, Graph von $f^{-1}$?
$\Gamma_f = \{(x, f(x)): x \in X\}$\\
$\Gamma_{f^{-1}} = \{(f(x), x)\}$\\
entsteht aus $\Gamma_f$ durch Spiegelung an der Hauptdiagonalen $x = y$\\
\begin{tikzpicture}[domain=-2:2, scale=0.5, prefix="plots/", smooth]
    \draw[very thin,color=gray] (-2.1,-2.1) grid (3.1,3.1);
    \draw[->] (-2,0) -- (3.2,0) node[right] {$x$};
    \draw[->] (0,-2.1) -- (0,3.2) node[right] {$y$};
    \draw[color=red, domain=-2:3] plot[id=v06g4] function{log(x)} node[right] {$log(x)$};
    \draw[color=green, domain=-2:1.35] plot[id=v06g5] function{exp(x)} node[right] {$exp(x)$};
\end{tikzpicture}