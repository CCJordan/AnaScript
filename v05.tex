% Kopfzeile beim Kapitelanfang:
\fancypagestyle{plain}{
%Kopfzeile links bzw. innen
\fancyhead[L]{\calligra\Large Vorlesung Nr. 5}
%Kopfzeile rechts bzw. außen
\fancyhead[R]{\calligra\Large 28.10.2013}
}
%Kopfzeile links bzw. innen
\fancyhead[L]{\calligra\Large Vorlesung Nr. 5}
%Kopfzeile rechts bzw. außen
\fancyhead[R]{\calligra\Large 28.10.2013}
% **************************************************
%
\section{Definition: Supremum, Infimum}
\enum{
	\item $s\in \R$ heißt \ul{Supremum} von $M$ $\Leftrightarrow$ $s$ ist kleinste obere Schranke von $M$. \\
	Das heißt:
	\enum{
		\item $s$ ist obere Schranke von $M$
		\item $s≤s'$ für jede weitere obere Schranke $s'$ von $M$ 
	}
	\item $t \in \R$ heißt \ul{Infimum} von $M$ $\Leftrightarrow$ $t$ ist größte untere Schranke von $M$.
}
1.1 zeigt: $M$ hat höchstens ein Supremum $s$: Bezeichnung $s = sup(M)$\\
Denn: $s, s'$ seien Suprema $\Rarr$ $s≤s' \wedge s'≤s \Rarr s = s'$\\
Analog $M$ hat höchstens ein Infimum. Bezeichnung: $t = inf(M)$
\section{Vollständigkeitsaxiom (Supremumeigenschaft von $\R$)}
Sei $M \subseteq \R, M \neq \emptyset$ nach oben beschränkt $\Rarr M$ besitzt ein Supremum.
\section{Folgerung}
Sei $M \subseteq \R, M \neq \emptyset$  nach unten beschränkt, so besitzt $M$ ein Infimum. 
\desc{Beachte}{Hat $M$ ein Maximum $m = max(M)$ so ist zugleich $m = sup(M)$}
\section{Lemma}
Sei $M \subseteq \R, M \neq \R$ nach oben beschränkt, $s = sup(M)$\\
$\forall \e > 0 \exists x \in M: s - \e < x ≤ s$
\subsubsection*{Beweis}
$s-\e$ ist keine obere Schranke von M, da s kleinste obere Schranke. $\Rarr \exists x\in M: s-\e < x$\\
Dabei $x ≤ s$ da $s$ obere Schranke von M. $\qed$
\section{Satz: Archimedische Eigenschaft}
(AR) $\forall x \in \R \underbrace{\exists n \in \N}_{\text{Abhängig von } x}: n > x$
\subsubsection*{Beweis}
Angenommen die Archimedische Eigenschaft (AR) gilt nicht, d.h. $\exists x \in \R: x ≤ n$ für alle $n \in \N$\\
$\Rarr \N$ ist durch x nach oben beschränkt.\\
$s = sup(M)$ existiert.\\
$\underset{\text{Lemma 3.13}}{\Rarr} \exists n \in \N: s-1 < n $\\
Widerspruch gegen Supremum Definition.
\section{Folgerungen aus der Archimedischen Eigenschaft}
\enum {
	\item $\forall \e > 0 \exists n \in \N$: $\frac{1}{n} < \e$
	\item Wachstum von Potenzen\\
	Sei $a \in \R, a > 1 \Rarr \forall M > 0 \exists n \in \N: a^n >M$
	\item Sei $a\in \R, 0 < a < 1 \Rarr \forall \e > 0\exists n \in \N:a^n < \e$
}
\subsubsection*{Beweise}
\enum {
	\item Sei $\e > 0$ (AR) $\to \exists n \in \N: n > \frac{1}{\e} \Rarr \frac{1}{n} < \e$
	\item Sei $M > 0$ $x:= a-1 > 0 \Rarr a^n = (1+x)^n ≥ 1 + x \cdot n$\\
	(AR) $\exists n \in \N: n > \frac{M}{x} \Rarr a^n > M$\\
	\item Sei $\e > 0. \frac{1}{a}>1 \Rarr \exists n\in \N: \left(\frac{1}{a}\right)^n>\frac{1}{\e} \to a^n < \e \qed$\\
	Man kann zeigen: Es gibt (bis auf Umbenennungen) genau einen angeordneten Körper, der das Vollständigkeitsaxiom erfüllt, nämlich $\R$.
}
$\forall \e > 0 \exists \nN \frac{1}{n} < \e$
\section{Satz: Existenz von Wurzeln}
Sei $a\in \R, a ≥ 0$ und $k \in \N\Rarr \exists! x \in \R, x≥0: x^k = a$\\
Schreibweise: $x= a^{\frac{1}{k}} = \sqrt[k]{a}$ $k$-te Wurzel aus $a$\\
$\sqrt{a} := \sqrt[2]{a}$\\
\desc{Beachte}{$\sqrt[k]{0} = 0$ für $a > 0$ ist $\sqrt[k]{a} > 0$(per Definition)}
\subsubsection*{Beweis}
\enum{
	\item Eindeutigkeit:\\
	Seien $x_1, x_2 ≥ 0 x_1 \neq x_2$ mit $x_1^k = a = x_2^k$\\
	Sei etwa $x_1 < x_2 \underset{\Rarr}{3.4.8} x_1 < x_2$
	\item Existenz:\\
	Hier für $k = 2$ ($k > 2$ ähnlich, aber aufwendiger)\\
	Sei o.E. (ohne Einschränkungen) $a ≥ 1$\\
	Setze $M:=\{y>0: y^2 ≤ a\}$\\
	$M \neq \emptyset$ (da $1 \in M$). $M$ ist nach oben durch $a$ beschränkt.\\
	Angenommen es gebe $y \in M: y > a \Rarr y^2 > a^2 \underset{≥}{a≥1} a\lightning$\\
	Setze $x:= suo(M)$. Behauptung: $x^2 = a$\\
	Annahme 1:\\
	$x^2 > a \Rarr \e := \frac{x^2 -a}{2x} > 0$\\
	Lemma 3.13 $\Rarr$ $\exists y \in M: x - \e < y ≤ x$ Dabei $y^2 ≤ a$\\
	$x^2 - a ≤ x^2 - y^2 = (x+y)(x-y) < 2x\e - x^2 -a $ \lightning
	\\
	Annahme 2:\\
	$x^2 < a \Rarr \frac{a}{x^2} > 1 \Rarr \frac{a}{x^2} - 1 > 0$\\
	Sei $r:= min(1, \frac{1}{3}\left(\frac{a}{x^2} - 1\right)) > 0$\\
	$\Rarr (1+r)^2 = 1 + (2+r) \cdot r < 1 + 3r ≤ \frac{a}{x^2}$\\
	$\Rarr [x(1+r)^2] ≤ a$\\
	$\Rarr x(1+r) \in M$ aber: $x(1+r) > x$
}
\section{Rechenregel zu Wurzeln}
Seien $x, y \in \R, x, y> 0$  und $k\nN\Rarr\sqrt[k]{x}\cdot \sqrt[k]{y} = \sqrt[k]{x\cdot y}$
\subsubsection*{Beweis:}
$(\sqrt[k]{x} \cdot \sqrt[k]{x})^k = (\sqrt[k]{x})^k \cdot (\sqrt[k]{x})^k = x \cdot y \Rarr$ Behauptung.\\
Wir wissen: $\sqrt{2} \in \R \setminus \Q$. Also $\Q \subseteq \R$\\
Die Zahlen aus $\R \setminus \Q$ heißen irrational.
\bem
Im Körper $\Q$ ist das Vollständigkeitsaxiom \ul{nicht} erfüllt, das heißt, es gibt Mengen $M \subseteq \Q, M\neq \emptyset, M$ beschränkt, so dass $M$ kleine kleinste obere Schranke im $\Q$ hat.\\
Denn: sonst würde Satz 3.16 auch im $\Q$ gelten $\Rarr x^2 = 2$ halt eine Lösung in $\Q$ \lightning
\chapter{Funktionen}
\section{Definition}
Seien $X, Y$ Mengen, eine \ul{Abbildung} $f$ von $X$ nach $Y$ ist eine Vorschrift, die jedem $x \in X$ genau ein $y\in f(x) \in Y$ zuordnet.\\
Schreibweise: $f: X \to Y, x \mapsto f(x)$\\
\begin{description}
	\item[X] Definitionsbereich von $f$
	\item[Y] Zielbereich von $f$
	\item[f(x)] Bild von $x$ unter $f$
\end{description}
\subsection*{Graph von $f$}
$\Gamma_f = \{x, \}$