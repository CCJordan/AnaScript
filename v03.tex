% Kopfzeile beim Kapitelanfang:
\fancypagestyle{plain}{
%Kopfzeile links bzw. innen
\fancyhead[L]{\calligra\Large Vorlesung Nr. 3}
%Kopfzeile rechts bzw. außen
\fancyhead[R]{\calligra\Large 21.10.2012}
}
%Kopfzeile links bzw. innen
\fancyhead[L]{\calligra\Large Vorlesung Nr. 3}
%Kopfzeile rechts bzw. außen
\fancyhead[R]{\calligra\Large 21.10.2013}
% **************************************************
%
\section{Der Binomische Satz}
Seien $x, y \in \R, n \in \N_0$
$$(x+y)^n = \sum_{k=0}^{n} \begin{pmatrix}n\\k\end{pmatrix} x^k y ^{n-k}$$
\subsection*{Beweis}
Mit Induktion nach $n$:
\enum{
	\item[I] Induktionsanfang $n=0$:\\
	$(x + y)^0 = 1 = \begin{pmatrix}0\\0\end{pmatrix} x^0y^0$
	\item[II] Induktionsschritt $n\to n + 1$:\\
	$(x+y)^{n+1} = (x+y)^1 + (x+y)^n \underset{I.V.}{=} (x+y)^1 \cdot \left( \sum_{k=0}^{n} \begin{pmatrix}n\\k\end{pmatrix} x^k y ^{n-k} \right)$\\
	$\Rarr x\cdot (x) + y \cdot (x)$\\
	$\Rarr \sum_{k=0}^{n} \begin{pmatrix}n\\k\end{pmatrix} x^{(k + 1)} y ^{n-k} + \sum_{k=0}^{n} \begin{pmatrix}n\\k\end{pmatrix} x^k y ^{n-k + 1}$\\
	$\underset{Indexverschiebung}{\Rarr}\sum_{k=1}^{n+1} \begin{pmatrix}n\\k\end{pmatrix} x^k y ^{n-(k-1)}$
	$\Rarr (x+y)^{n+1} = y^{n+1} + x^{n+1} + \sum_{k=1}^{n} \left[\begin{pmatrix}n\\k\end{pmatrix} + \begin{pmatrix}n\\k-1\end{pmatrix}\right] x^k y ^{n+1-k}$\\
	$\Rarr \sum_{k=0}^{n+1} \begin{pmatrix}n+1\\k\end{pmatrix} x^k y ^{n-(k-1)}$ \qed
}
\chapter{Die Reellen Zahlen}
Ein Ziel bei der Erweiterung von Zahlenbereichen ist die Lösbarkeit von Gleichungen.\\
$\N \to \Z: x + n = m \qquad (n,m \in \N_0)$\\
$\Z \to \Q: x \cdot n = m \qquad (n,m \in \Z, n \neq 0)$\\
Aber $x^n = m (n,m \in \N)$ hat im Allgemeinen keine Lösungen im $\Q$
\section*{Axiomatische Einführung von $\R$}
Wir geben eine Reihe von Grundlegenden Eigenschaften von $\R$ an.
\enum{
	\item Körperaxiome
	\item Anordnungsaxiome
	\item Vollständigkeitsaxiom
}
Man kann zeigen (wichtiger Satz): Es gibt genau 1 Menge $\R$ mit diesen Eigenschaften.\\
Es gibt präzise Konstruktionen\\
$\N \to \Z \to \Q \underset{\text{Vervollständigung}}{\longrightarrow} \R$ so dass $\N \subseteq \Z \subseteq \Q \subseteq \R$
\section{Vorbemerkung: Quantoren}
Gegeben: Aussagen $P(x), x\in X (X \text{ sei Menge})$\\
\begin{tabular}{l|l}
Aussage & Schreibweise\\\hline
Für alle $x\in X$ gilt $P(X)$ & $\forall x\in X: P(x)$\\
Es gibt (mindestens) ein x mit $P(x)$ & $\exists x \in X: P(x)$\\
Es gibt genau ein $x \in X$ mit $P(x)$ & $\exists! x \in X: P(x)$\\
Es gibt kein $x \in X$ mit $P(x)$ & $\not\exists x\in X: P(x)$
\end{tabular}\\
$\forall: Allquantor, \exists: Existenzquantor$
\section*{I. Körperaxiome}
Auf der Menge $\R$ sind zwei Rechenoperationen + (Addition) und $\cdot$ (Multiplikation) erklärt, sodass $(\R, +, \cdot)$ ein Körper ist.
\section{Definition: Körper}
Ein Körper ist eine Menge $\mathbb{K}$ mit zwei Operationen + (Addition) und $\cdot$ Multiplikation, sodass gilt:
\enum{
	\item $\forall x, y \in \mathbb{K}$:\\
	$(x+y) + z = x +(y+z)$ (Assoziativität für +)\\
	$(x\cdot y) \cdot z = x \cdot (y \cdot z)$ (Assoziativität für $\cdot$)
	\item $\forall x,y \in \mathbb{K}$:\\
	$x+y = y + x$ (Kommutativität für +)\\
	$x \cdot y = y \cdot x$ (Kommutativität für $\cdot$)\\
	\item Existenz neutraler Elemente:\\
	$\exists 0 \in \mathbb{K}$ (Null) mit: $x + 0 = x$\\
	$\exists 1 \in \mathbb{K}$ (Eins) mit: $x \cdot 1 = x$
	\item Existenz von Inversen:\\
	$\forall x \in \mathbb{K} \exists y \in \mathbb{K} x + y = 0$ (additives Inverses von x)\\
	$\forall x \in \mathbb{K}\setminus \{0\} \exists z \in \mathbb{K} x + z = 1$ (multiplikatives Inverses von x)
	\item Distributivgesetz\\
	$\forall x,y,z \in \mathbb{K}: x \cdot (y + z) = x y + x z$
}
\section{Folgerungen}
Sei $K$ ein Körper.
\enum{
\item 0 und 1 sind eindeutig bestimmt.\\
Beweis für 0 (für 1 analog):\\
Sei 0' neutral bezüglich +, sodass $0 = 0 + 0' = 0' + 0 = 0'$
\item $y, z$ in (4) sind (bei festem x) eindeutig bestimmt:\\
Beweis für y: \\
Sei $x + y = 0 = x + y $\\
$\Rarr y = y + 0 = y + (x + y') = (y + x) + y' = (x + y) + y' = y'$\\
Bez. $-x$: additives Inverses\\
$x^{-1} = \frac{1}{x}$
\item $a, b\in \mathbb{K} \Rarr$ die Gleichung $a + x = b$ hat eine ein eindeutige Lösung, nämlich x = b+ (-a)
}