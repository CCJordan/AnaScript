\documentclass[12pt]{scrreprt} % scrartcl
\usepackage[utf8]{inputenc}
\usepackage{a4wide}
\usepackage{ulem,lmodern,dsfont}
\usepackage[left=2cm,right=2cm,top=2cm,bottom=2.5cm]{geometry}
%\pagestyle{empty} %keine Seitenzahlen
\usepackage{graphics}
\usepackage{fontspec}
\usepackage{calligra}
\usepackage[ngerman]{babel} % Deutsche Formate
\usepackage{blindtext}
\usepackage{stringstrings} % Substrings  etc.

\usepackage{amsmath}	% Mathebibliothek 2
\usepackage{amssymb}	% Mathebibliothek 2
\newcommand{\qed}{\hfill\text{\calligra q.e.d.}}
\newcommand{\C}{\mathds{C}}%komplexe Zahlen
\newcommand{\R}{\mathds{R}}%reelle Zahlen
\newcommand{\Q}{\mathds{Q}}%rationale Zahlen
\newcommand{\Z}{\mathds{Z}}%ganze Zahlen
\newcommand{\N}{\mathds{N}}%Natürliche Zahlen
\newcommand{\Lsg}{\mathds{L}}%Lösungsmenge
\newcommand{\F}{\mathds{F}}%Körper (z.B. F_2)
%Zahlenmengen end
\newcommand{\rem}[1]{} %Blockweise kommentieren ;)
\begin{document}
Entwikelte Formel:\\
$f(n) = \frac{n}{n+1}$\\
I.A.: $n = 1$\\
$f(n) = \frac{1}{2}$
\hfill\checkmark
\\
I.S.: $n \to n+1$\\
$\frac{n+1}{n+1} = \frac{1}{1 \cdot 2} + \frac{1}{2 \cdot 3} + \cdots + \frac{1}{(n+1)(n+2)}$\\
$\overset{I.V.}{\Leftrightarrow} \frac{n+1}{n+2} - \frac{n}{n+1} = \frac{1}{(n+1)(n+2)}$\\
$\Leftrightarrow \frac{(n+1)(n+1) - (n(n+2))}{(n+1)(n+2)} = \frac{1}{(n+1)(n+2)}$\\
$\Leftrightarrow (n+1)(n+1) - (n(n+2)) = 1$\\
$\Leftrightarrow (n^2+2n+1) - (n^2+2n) = 1$\\
$\Leftrightarrow 1 = 1$
\hfill{\calligra q.e.d.}
\end{document}