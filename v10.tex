% Kopfzeile beim Kapitelanfang:
\fancypagestyle{plain}{
%Kopfzeile links bzw. innen
\fancyhead[L]{\calligra\Large Vorlesung Nr. 10}
%Kopfzeile rechts bzw. außen
\fancyhead[R]{\calligra\Large 14.11.2013}
}
%Kopfzeile links bzw. innen
\fancyhead[L]{\calligra\Large Vorlesung Nr. 10}
%Kopfzeile rechts bzw. außen
\fancyhead[R]{\calligra\Large 14.11.2013}
% **************************************************
%
\subsection{Beweis zu 3.}
$z = x+iy\ , w = u+iv \Rarr \bar{zw} = \bar{(x+iy) (u+iv)} = \bar{xu - yv + i(yu + xv)} = xu - yv + i(yu + xv) = (x-iy)(u-iv)$
\section{Definition: Betrag von $z = x+iy$}
$|z| = \sqrt{x^2 + y^2} = z\bar{z}$
\section{Eigenschaften des Betrags}
\enum{
	\item $|z| ≥ 0, |z| = 0 \Leftrightarrow z = 0$
	\item $|\bar{z}| = |z|$
	\item $|Re(z)| ≤ |z|, |Im(z)| ≤ |z|$
	\item $|z\cdot w| = |z|\cdot |w|$
	\item $|z+w|≤|z|+|w|$ Dreiecksungleichung
	\item $||z|-|w|| ≤ |z - w|$
}
\subsection*{Beweis zu 4:}
$|z\cdot w|^2 = z\cdot w \cdot \bar{z\cdot w} = z\cdot \bar{z} \cdot w \cdot \bar{w} = |z|^2 \cdot |w|^2$\\
Wurzelziehen $\Rarr |zw| = |z|\cdot |w|$
\subsection*{Beweis zu 5:}
$|z+w|^2=(z+w)(\bar{z+w}) = (z+w)(\bar{z} + \bar{w}) = z\bar{z} + (z\bar{w} + \bar{z} + w)+ w\bar{w}$\\
$=|z|^2 + 2 Re(z\cdot \bar{w}) + |w|^2 ≤ (|z|+|w|)^2$
\subsection*{Beweis zu 6:}
6 folgt aus 5 wie für $\R$
\bem
\enum{
	\item Sei $a\in \C, r > 0 \Rarr \{z\in \C: |z-a|=r\}$
	\item Zum invertieren von $\C$:\\
	Sei $z = x + iy \in \C, z \neq 0$\\
	$\Rarr \frac{1}{z} \frac{\bar{z}}{z\cdot \bar{z}} = \frac{x - iy}{x^2 + y^2} = \frac{x}{x^2 + y^2} - i \frac{y}{x^2 + y^2}$
	\item Es gibt keine Anordnung ">" auf $\C$ bezüglich der komplexen Zahlen.\\
	Denn, sonst wäre $i^2 = -1 > 0 \Rarr 1 < 0$ \lightning\\
	Aber $1 > 0$ in jedem angeordneten Körper.
	$z>w, z>0$ macht in $\C$ keinen Sinn, nur die Beträge der komplexen Zahlen kassen sich der Größe nach vergleichen.\\
	Geometrische interpretation von $+, \cdot$\\
	\begin{description}
		\item[Addition] $z=x+iy, w=u+iv \Rarr z + w = (x + u) + i(y + v) $\\
		Vektoraddition in $\R^2$
		\item[Multiplikation] $z, w \in \C \Rarr |zw| = |z| \cdot |w|$\\
		Wir werden sehen: $zw$ entsteht z, w durch Multiplikation der Beträge und Addition der Winkel zur $\R$-Achse.
	\end{description}
}
\section*{Quadratische Gleichungen in den komplexen Zahlen}
Seien $p, q \in \C$ Betrachte die Gleichung:
$(*) z^2 + pz + q = 0$\\
Lösung in $\C$?\\
$(*) \Leftrightarrow \left(z + \frac{p}{2}\right)^2 = -q + \frac{p^2}{4} =: C$\\
Sei $c = a + ib \in \C (a, b \in \R)$\\
Gesucht: $z \in \C$ mit $z^2 = c$ \\
Mit $z$ ist dann auch $-z$ eine Lösung.\\
Ansatz: $z = x +iy (x, y \in \R)$\\
$z^2 = c \Leftrightarrow (x+ iy)^2  = a + ib$\\
$\Leftrightarrow x^2 + 2ixy -y^2 = a+ib$\\
$\Leftrightarrow x^2-y^2 = a \wedge 2xy = b$\\
Vergleich der Real- und Imaginärteile.\\
Beachte: 2 komplexe Zahlen sind genau dann gleich, wenn ihre Real- und Imaginärteile gleich sind. Auflösen nach $x, y$ liefert 2 komplexe Lösungen. $z_1 = x_1 + iy_1, z_2 = -z_1$\\
Schreibe: $z_{1,2} = ±\sqrt{c}$\\
Damit hat $(*)$ die Lösungen $z_{1/2} = -\frac{p}{2} ± \sqrt{\frac{p^2}{4} -q}$\\
\section*{Algebraische Gleichungen in den komplexen Zahlen}
Eine algebraische Gleichung in $\C$ ist eine Gleichung der Form 
$$z^n + a_{n-1}z^{n-1} + ... + a_1 + a_0 = 0$$
mit Koeffizienten $a_0, ... a_{n-1} \in \C, n\in \N$ und unbekannter $z\in \C$
\section{Satz: Fundamentalsatz der Algebra}
Jede algebraische Gleichung hat mindestens eine Lösung im $\C$\\
(Hier ohne Beweis)
\subsection*{Polynome}
Hier $\mathbb{K}$ einer der Körper $\R$ oder $\C$.
\subsection*{Vorbemerkung}
Seien $f, g: X \to \mathbb{K}$ Funktionen (X Menge, z.B. $\R, \C$)
Definiere $f±g$, $f\cdot g: X \to \mathbb{K}$\\
$(f±g)(x) = f(x) ± g(x)$, $(f\cdot g)(x)= f(x) \cdot g(x)$\\
Ferner $\frac{f}{g}:\{x\in X: g(x) \neq 0 \forall x \to \mathbb{K}\} \Rarr\left(\frac{f}{g}\right)(x) \frac{f(x)}{g(x)}$
\section*{Definition}
Eine Funktion $p:\mathbb{K} \to \mathbb{K}$ der Form $p(x) = a_nx^n + a_{n-1}x^{n-1} + ... + a_1x + a_0$\\
Mit $n\in \N, a_0, ..., a_n \in \mathbb{K}$ heißt Polynomfunktion, kurz Polynom. über $\mathbb{K}$\\
Ist $a_n \neq 0$ so heißt $n$ der \ul{Grad} von $p$.\\
$n = grad(p)$\\
Sind alle $a_i = 0$ so heißt $p$ Nullpolynom.\\
$grad(0) = -∞$\\
$\mathcal{P}_{\mathbb{K}} := \{p:\mathbb{K} \to \mathbb{K}: p \text{ Polynom}\}$\\
Summen und Produkte von Polynomen sind wieder Polynome.\\
\section{Satz: Polynomdivision mit Rest}
Seien $p. q \in \mathcal{P}_{\mathbb{K}}, q = 0$\\
Es gibt eindeutige $r, s \in \mathcal{P}_{\mathbb{K}}$ mit\\
$p = sq + r$ wobei $grad(r) < grad(q)$
(Nehmen Sie es hin. Beweis evtl. in LinA.)
\subsection*{Bezeichnung}
Ist $r = 0$ in 6.10. das heißt, $p = sq$, so heißt $q$ ein Teiler von $p$\\
$\alpha \in \mathbb{K}$ heißt \ul{nullstelle} von $p in \mathcal{P}_{\mathbb{K}}$ falls p(x)=0
\section{Korrolar: Abspalten von Linearfaktoren}
Sei$ p(x) = 0 \Rarr \exists! s \in \mathcal{P}_{\mathbb{K}}$ $p(x) = (x - \alpha) \cdot s(x)$ dabei gilt $grad(s) = grad(p) -1$