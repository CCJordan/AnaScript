\section{Beispiele} Ausgelassen
\section*{Sätze über stetige Funktionen}
Erinnerung $f:D\to \R$ stetig in $x_0 \in D \Leftrightarrow \forall \e > 0 \exists \delta > 0: |f(x) - f(x_0)|< \e\forall x \in D$ mit $|x-x_0|<\delta$
\section{Zwischenwertsatz}
Seien $a, b \in \R$ mit $a<b$ und $f:[a, b] \to \R$ stetig $\Rarr f$ nimmt jeden Wert zwischen $f(a)$ und $f(b)$ an.\\
%todo schaubild
Das heißt: $\gamma$ zwischen $f(a)$ und $f(b) \Rarr \exists x \in [a, b]: f(x) = \gamma$\\
\subsection*{Beweis}
Sei etwa $f(a) ≤ f(b)$ ohne Einschränkung $f(a) < \gamma < f(b)$\\
$M:= \{t\in [a,b]:f(t) ≤ \gamma\}$\\
$M≠\emptyset$ (da $a\in M$), beschränkt $\Rarr x:= sup(M)$ existiert, $x\in [a, b]$\\
\ul{Behauptung$ f(x) = \gamma$}\\
Dazu $x= sup(M) \Rarr \exists \text{Folge} (t_n) \subseteq M: x- \frac{1}{n} < t_n ≤ x$\\
$\lim\limits_{n\to ∞} t_n = x; f(t_n) ≤ \gamma$\\
$f$ stetig $\Rarr f(x) = \lim f(t_n) ≤ \gamma < f(b)$ insbesondere $x<b$\\
Wähle Folge $(s_n)$ in (x,b] mit $s_n \to x \Rarr f(s_n) > \gamma (s_n \notin M)$ und $f(x) = \lim f(s_n) ≥ \gamma$ (Vergleichskriterium 5.8)\\
Zus: $f(x) = \gamma$
\section{Satz}
Sei $p \in \P_\R$ reelles Polynom ungeraden Grades $\Rarr p$ hat mindestens eine reelle Nullstelle\\
\subsection*{Beweis}
$p$ habe o. E. Leitkoeffizienten = 1, d.h. $p(x)= x^n + a_{n-1}x^{n-1} + ... + a_0$\\
$\underset{x≠0}{=}x^n\underbrace{\left( 1 + \frac{a_{n-1}}{x} + ... + \frac{a_0}{x_n}\right)}_{=:h(x)}$\\
$\lim\limits_{x\to ±∞} h(x) = 1$\\
$\Rarr \exists s > 0: h(s) > 0$\\
$\Rarr \exists r < 0: h(r) > 0$\\
$\Rarr p(s) = s^n \cdot h(s)>0; p(r) = r^n \cdot h(r) < 0$\\
ZWS $\Rarr \exists x \in [r, s]: p(x) = 0 \qed$\\
\subsection*{Bezeichnungen}
\enum{
	\item Ein angeschlossenes beschränktes Intervall $[a, b] \subseteq \R$ heißt auch \ul{kompakt}.
	\item Eine Funktion $f:D\to \R$ heißt beschränkt $\Leftrightarrow \exists M >0 |f(x)|≤ M \forall x \in D$
}
\section{Satz von Maximum und Minimum}
Sei $[a,b] \subseteq \R$ kompaktes Intervall, $f: [a,b] \to \R$ stetig $\Rarr f$ beschränkt und $\exists x_1, x_2 \in [a, b]: \forall x \in [a, b]$ gilt:\\
$f(x_1) ≤ f(x) ≤ f(x_2)$
\subsection*{Beweis}
Nachweis des Maximums (Minimum analog)\\
$s:= sup(f(x):x\in [a,b]) \in \R \cup \{+∞\}$\\
$s:= +∞$ falls $f$ nach oben unbeschränkt\\
Definition von $sup \Rarr \exists $ Folge $(x_n) \subseteq [a,b]: f(x_n) \to s$\\
$(x_n)$ beschränkt \Rarr (Bolzano-Weierstraß) $(x_n)$ hat eine konvergente Teilfolge $(x_{n_k})$\\
Sei $x_{n_k} \to x\in \R$ $a≤ x_{n_k} ≤ b \forall k \Rarr a≤x≤b$\\
$f$ stetig $\Rarr f(x_{n_k}) \to f(x) \Rarr s = f(x) < ∞$ und $s = max(f(x)) x\in [a,b]$\\
\chapter{Monotone Funktionen und ihre Umkehrfunktionen}
\section{Definition}
Sei $D\subseteq \R; f: D \to \R$ heißt monoton wachsend [fallend] $\Leftrightarrow \forall x, y\in D$ mit $x < y$ gilt $f(x) ≤[≥] f(y)$\\
Ist sogar stets $f(x) <[>] f(y)$ so heißt $f$ streng monoton wachsend [fallend].
\section{Lemma}
Sei $f: D\to \R$ streng monoton wachsend [fallend] $\Rarr f: D \to f(D) = \{f(x) : x\in D\}$ ist bijektiv und $f^{-1} : f(D) \to D$ ist streng monoton wachsend [fallend]\\
\subsection{Beweis}
$x<y \Leftrightarrow f(x) < f(y)$ (*)\\
(zu "$\Leftarrow$" $x ≥ y \underset{s.m.w.}{\Rarr} f(x) ≥ f(y) \lightning$)\\
$\Rarr f$ injektiv, also $f:D\to f(D)$ bijektiv und aus $\Leftarrow$ in (*)$f^{-1}$ s.m.w.\\
\section{Satz}
Sei $f: [a, b]\to \R$ streng monoton wachsend und stetig $\Rarr f:[a, b] \to [f(a), f(b)]$ bijektiv und $f^{-1}$ ist ebenfalls streng monoton wachsend und stetig.\\
Sei $y_0 \in [f(a), f(b)], y_0 = f(x_0)$\\
Hier: Fall $a< x_0 < b$ ($x_0 \in \{a, b\}$ analog)\\
Sei $\e > 0$ gegeben, o.E. so klein, dass $(x_0 - \e, x_0 + \e) \subseteq [a, b]$\\
$\alpha = f(x_0 - \e), \beta := f(x_0 + \e)$\\
$\alpha < y_0 < \beta$ ($f$ streng monoton wachsend)\\
Sei $y \in [\alpha, \beta] \Rarr f^{-1}(\alpha) ≤ f^{-1}(y) ≤ f^{-1}(\beta)$\\
$\Rarr |f^{-1} (y) - f^{-1}(y_0)|< \epsilon \forall y\in [\alpha, \beta]$\\
$\Rarr f^{-1}$ stetig in $y_0$
\section{Beispiele}
\ul{Wurzelfunktion}\\
Sei $k \in \N$ fest.\\
Potenzfunktion: $f(x) = x^k, [0, ∞) \to [0, ∞)$\\
$f$ stetig und streng monoton wachsend, f auch bijektiv, denn: $y ≥ 0 \Rarr \exists! x≥ 0: x^k = y$ nämlich $x = \sqrt[k]{y}$\\
Umkehrfunktion: $f^{-1} = x\mapsto \sqrt[k]{x}, [0,∞) \to [0,∞)$ k-Wurzelfunktion