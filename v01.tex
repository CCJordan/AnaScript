% Kopfzeile beim Kapitelanfang:
\fancypagestyle{plain}{
%Kopfzeile links bzw. innen
\fancyhead[L]{\calligra\Large Vorlesung Nr. 1}
%Kopfzeile rechts bzw. außen
\fancyhead[R]{\calligra\Large 14.10.2013}
}
%Kopfzeile links bzw. innen
\fancyhead[L]{\calligra\Large Vorlesung Nr. 1}
%Kopfzeile rechts bzw. außen
\fancyhead[R]{\calligra\Large 14.10.2013}
% **************************************************
%
\chapter{Logik und Mengen}
\section*{Aussagen}
Eine Mathematische Aussage ist ein sprachlicher Ausdruck, dem genau einer der Wahrheitswerte "wahr" (w) oder "falsch" (f) zugeordnet werden kann. Eine Aussage im Mathematischen Sinne stellt eine objektive Tatsache dar.
\bsp
\textit{5 = 3 + 2} ist eine wahre Aussage.\\
\textit{2 $\cdot$ 3 = 9} ist eine falsche Aussage.\\
\textit{Das Essen ist gut gewürzt.} ist gar keine Aussage, da es sich hierbei nicht um ein objektive Faktum handelt.
\section{Verknüftung von Aussagen}
Liefert neue Aussagen. Seien $P, Q$ Aussagen.
\begin{enumerate}
\item \underline{Konjunktion}: $P \wedge Q$ ($P$ und $Q$)\\
Wahr genau dann, wenn P und Q wahr sind.
\item \underline{Disjunktion}: $P \vee Q$ ($P$ oder $Q$)\\
Wahr genau dann, wenn mindestens eine der Aussagen wahr ist.
\item \underline{Negation}: $\neg P$\\
Wahr g.d.w. $P$ falsch
\item \underline{Implikation}: $P \to Q$ (P impliziert Q, aus P folgt Q)\\
falsch g.d.w. P wahr und Q falsch, sonst wahr.\\
Bemerkung: Ist P falsch, so ist $P \to Q$ stets wahr.
\item \underline{Äquivalenz}: $P \leftrightarrow Q$ :=  $P \to Q \wedge Q \to P$ ($P$ äquivalent zu $Q$)\\
wahr g.d.w. $P$ und $Q$ den gleichen Wahrheitswert haben.
\end{enumerate}
% Wahrheitstabelle einfügen.
\section{Bezeichnung: 'semantische Äquivalenz'}
Zwei Aussagen $P, Q$, welche die gleiche Wahrheitstabelle haben, heißen semantisch äquivalent $P \equiv Q$\\
Bsp.
\begin{enumerate}
\item $\neg(\neg P) \equiv P$ denn % Wahrheitstabelle
\item De Morgan'sche Gesetze:\\
$\neg (P \wedge Q) \equiv \not P \vee \neg Q$\\
$\neg (P \vee Q) \equiv \not P \wedge \neg Q$
\item Distributivgesetze:\\
$P \wedge (Q \vee R) \equiv (P \wedge Q) \vee (P \wedge R)$\\
$P \vee (Q \wedge R) \equiv (P \vee Q) \wedge (P \vee R)$
\end{enumerate}
\subsection*{Mengen}
Definition nach G. Cantor Eine Menge ist eine Zusammenfassung bestimmter wohlunterschiedener Objekte unseres Denkens zu einem Ganzen. Diese Objekte heißen die Elemente der Menge.\\
Diese Definition ist für mathematische Verhältnisse vage.\\
Pragmatischer Standpunkt: Eine Menge ist gebildet, wenn feststeht welche Elemente dazugehören.\\
\subsection*{Schreibweisen}
$x \in A$ heißt $x$ ist Element der Menge $A$\\
$x \notin A$ falls $x$ nicht Element der Menge $A$\\
$A \subseteq B$ (A Teilmenge von B), falls jedes Element aus A auch in B liegt.\\
Entsprechend $B \supseteq A (B$ Obermenge von $A)$\\
\begin{itemize}
\item $A = B$ falls $A \subseteq B$ und $B \supseteq A$
\item strikte Inklusion $A \subsetneqq B$ A ist Teilmenge von $B$ g.d.w. ($A \neq B$)
\end{itemize}

\subsection{Beschreibung von Mengen}
\begin{enumerate}
\item Durch Aufzählung der Elemente\\
Bsp.: $A = \{Rot, Blau\}$, $A = \{1, \{1, 2\}, \{1, 2,\{1, 2\} \}\}$\\
$\mathbb{N} = \{1, 2, 3\}$ Menge der natürlichen Zahlen\\
$\mathbb{N}_0 = \{0, 1, 2, 3,...\}$\\
$\mathbb{Z} = \{0, \pm 1, \pm 2,...\}$
\item durch eine charakteristische Eigenschaft\\
$A = \{n \in \mathbb{Z}: N \text{ist gerade}\} = \{n \in \mathbb{Z}: \text{es gibt} k \in Z \text{ mit } n = 2k\}$
\end{enumerate}
\section{Definition Primzahlen}
Seien $n, m \in \mathbb{Z}$.\\
m heißt Teiler von n, (kurz $m\mid n$ ),falls ein $k \in \mathbb{Z}$ ist mit $n = k \cdot m$. Sonst $m \not\mid n$.
\section{Mengenoperationen}
Seien $A, B$ Mengen.
\begin{description}
\item[Vereinigung der Mengen $A$ und $B$]{$A \cup B := \{x: x \in A \vee x\in B\}$\\
Die Vereinigung aus $A$ und $B$ besteht aus den Elementen die in $A$ oder $B$ enthalten sind.}
\item[Durchschnitt der Mengen $A$ und $B$] {$A \cap B := \{x: x \in A \wedge x \in B\}$\\
Der Durchschnitt von $A$ und $B$ besteht aus den Elementen, die sowohl in $A$ als auch in $B$ enthalten sind.}
\item[Differenzmenge]{$A \setminus B := \{x: x \in A \wedge x \notin B\}$\\
Die Differenzmenge von $A$ und $B$ besteht aus jenen Elementen, die in $A$ jedoch nicht in $B$ enthalten sind.}
\end{description}
\section{Distributivgesetze}
\begin{enumerate}
\item $A \cap (B \cup C) = (A \cap B) \cup (A \cap C)$
\item $A \cup (B \cap C) = (A \cup B) \cap (A \cup C)$
%Beweis
\end{enumerate}
\section{Kartesisches Produkt von Mengen}
Seien $A, B$ Mengen.\\
$A \times B := {(a, b): a \in A, b\in B}$ Mengen aller geordneten Paare $(a,b)$ im Gegensatz zu $\{a, b\} = \{b,a\}$ kommt es bei $(a, b)$ auf die Reihenfolge an.
\section{Bezeichnung}
Sei A eine Menge.
\begin{enumerate}
\item $\mathcal{P}(A) =\{B: \subseteq A\}$ Potenzmenge von $A$
\item Ist $A$ endlich, so bezeichnet $\vert A \vert$ die Anzahl der Elemente von $A$ (Mächtigkeit von $A$)\\
Falls A unendlich schreibe (etwas lax) $\vert A \vert = \infty$
\end{enumerate}