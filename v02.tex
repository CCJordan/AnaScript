% Kopfzeile beim Kapitelanfang:
\fancypagestyle{plain}{
%Kopfzeile links bzw. innen
\fancyhead[L]{\calligra\Large Vorlesung Nr. 2}
%Kopfzeile rechts bzw. außen
\fancyhead[R]{\calligra\Large 17.10.2013}
}
%Kopfzeile links bzw. innen
\fancyhead[L]{\calligra\Large Vorlesung Nr. 2}
%Kopfzeile rechts bzw. außen
\fancyhead[R]{\calligra\Large 17.10.2013}
% **************************************************
\chapter{Natürliche Zahlen und vollständige Induktion}
Vollständige Induktion ist ein Beweisprinzip für Aussagen über natürliche Zahlen.\\
Es beruht auf dem:
\section{Induktionsaxiom}
Sei $M \subseteq \N$ mit
\enum{
\item $1 \in M$
\item Für alle $n \in \N$ gilt: $n \in M \Rarr n + 1 \in M$
}
Dann ist $M = \N$\\
\bem Axiome sind Grundlegende Aussagen in einer Theorie, die ohne Beweis vorausgesetzt werden.
\section{Satz: Prizip der vollständigen Induktion}
Zu jedem $n \in \N$ sei eine Aussage $A(n)$ gegeben. Es gelte
\begin{enumerate}
\item {A(1) ist wahr ( {Induktionsanfang})}
\item Für alle $n \in \N$ gilt: $A(n) \Rarr A(n+1)$
\end{enumerate}
Dann ist $A(n)$ wahr für alle $n \in \N$\\
"Dominoeffekt" $A(1)$ ist wahr $\Rarr A(2)$ ist wahr.\\
M erfüllt (1) und (2) das Induktionsaxiom. Also ist $M = \N$\\
\\
%
Verschiebung des Induktionsanfangs:\
Seien A(n) Aussage für $n \in \Z, n \geq n_0$ ($n_0 \in \Z$ fest)\\
Dann gilt das Prinzip der vollständigen Induktion entsprechend mit Induktionsanfang bei $n_0$ statt bei 1.
\section*{Summen und Produktzeichen}
Gegeben seien reelle Zahlen $a_n$ für $m \leq k \leq n$ wobei $m,n \in \Z, m \leq n$ Man setzt 
$$\sum_{k=m}^{n} a_k := a_m + a_{m+1} + ... + a_{n} $$
und 
$$ \prod_{k = m}^{n} a_k := a_m \cdot a_{m+1} \cdot ... \cdot a_n $$
Konvention für $m > n$:\\
$\sum_{k=m}^{n} a_k = 0$
und
$\prod_{k=m}^{n} a_k = 1$
\subsubsection*{Beispiel: Die Arithmetische Reihe}
$\sum_{k = 1}^{n} = 1 + 2 + ... n = ?$\\
Die Idee von Gauß für n = 100\\
$1 + ... 100 = (1 + 100) + (2 + 99) + ... (50 + 51) = 101 \cdot 50 = 5050$\\
\section{Satz}
$\sum_{k=1}^{n} k = \frac{1}{2} \cdot n \cdot (n+1)$ für alle $n \in \N$\\
Beweis mit vollständiger Induktion:\\
$A(n) = \sum_{k=1}^{n} k = \frac{1}{2} \cdot n \cdot (n + 1)$
% römische Zahlen
\begin{enumerate}
\item Induktionsanfang $n = 1$\\
A(1) ist wahr, denn $\sum_{k=1}^{1} k = \frac{1}{2} \cdot 1 \cdot 2$ \ok
\item Induktionsschluss $n \to n + 1$\\
Es sei A(n) wahr, das heißt, es gelte:\\
$\sum_{k = 1}^{n} = \frac{n (n+1)}{2}$ (Induktionsvoraussetzung)\\
$\Rarr \sum_{k = 1}^{n + 1} k = \sum_{k = 1}^{n} + (n + 1) \underset{I.V.}{=} \frac{1)}{2} \cdot n \cdot (n+1) + (n+1) = (n+1)(\frac{n}{2} + 1) = \frac{1}{2} (n+2) \cdot (n+1)$
\Rarr A(n+1) ist wahr. Mit vollständiger Induktion folgt die Behauptung.
\end{enumerate}
\bem zum Rechnen mit Summen (analog für Produkte):
$\sum_{K = 1}^{n + 1} a_k = \sum_{k = 1}^{n} a_k + a_{n+1}$\\
für $m \in \N$: $\sum_{k=1}^{m+n} a_k = \sum_{k = 1}^{n} a_k + \sum_{k = n + 1}^{m} a_k$
\subsection*{Beispiel 2: Geometrische Reihe}
Potenzen: Sei $x\in \R, n \in \N_0$ dann $x^n := x_1 \cdot x_2 \cdot ... \cdot x_n = \prod_{k=1}^{n} x$
Indexsumme $x^0 = 1$
\section{Satz: Gemetrische Summenformel}
Für $x \in \R\setminus \{1\}$ und $n \in \N_0$ gilt:
$$\sum_{k = 0}^{n} x^k = \frac{1 - x^{n+1}}{1 - x}$$
\subsubsection*{Beweis mit vollständiger Induktion}
% Römische Zahlen
\begin{enumerate}
\item Ind. Anfang $n = 0$: $x^0 = \frac{1 - x}{1 - x} = 1$
\item Ind. Schluss $n \to n + 1$\\
I.V.: $\sum_{k = 0}^{n} x^k = \frac{1 - x^{n+1}}{1 - x}$\\
\Rarr $\sum_{k = 0}^{n + 1} x^k = \sum_{k = 0}^{n} x^k + x^{n+1} = \frac{1 - x^{n+1}}{1 - x} + x^{n+1} 
= \frac{1 - x^{n+1} + (1-x)\cdot x^{n+1}}{1 - x} = \frac{1 - x^{n+2}}{1 - x}$
\end{enumerate}
\qed\\
Ein Beweis mit vollständiger Induktion setzt voraus, dass man die zu beweisende Identität bereits kennt, bzw. sie vermutet.\\
Eine solche Vermutung gewinnt man z.B. durch Berechnungen für kleine Werte von n.
\section{Definition: Fakultät}
Für $n \in \N_0$ setze $n! := \prod_{k=1}^{n} k = 1 \cdot 2 \cdot ... \cdot n$\\
Also $0! = 1$, $1! = 1$, $2! = 3$, $3! = 6$\\
$n! = (n+1)! \cdot n$\\
$n!$ wächst sehr schnell. z.B. $13! \approx 6,2 \times 10^9$\\
Es gibt dafür eine einfache Formel analog zur arithmetischen Summenformel.\\
\Satz
Die Anzahl der möglichen Anordnungen (auch Permutationen) der Elemente einer Menge M mit $|M| = n$ ist $n!$.
Beweis (bezeichne die Elemente 1, 2, ... n):
\enum{
\item Induktionsanfang: n = 1\\
1 = 1! eine Mögliche Anordnung. \ok
\item Induktionsschluss $n \to n + 1$\\
Besetze zunächst Position 1 dafür gibt es $n + 1$ Möglichkeiten.\\
Sei $P_k := \{$Permutationen von $1, 2, ... n + 1$ mit $k$ an Position 1\}\\
Nach I.V. ist $|P_k| = n!$ (Anzahl der Möglichkeiten die Stellen 2, 3, ... n+1 zu besetzen)
\Rarr Anzahl der Permutationen von $1, 2... (n+1) = (n+1)!$
}
\section{Definition: Binomialkoeffizienten}
Seinen $k, n \in \N_0$ $\begin{pmatrix}
n\\k
\end{pmatrix} := \prod_{j = 1}^{k} \frac{(n - j + 1)}{j}$\\
sprich "k aus n" oder "n über k"\\
Folgerungen:
\begin{enumerate}
\item $\begin{pmatrix}
n \\ 0
\end{pmatrix} = 1$,
$\begin{pmatrix}
n\\1
\end{pmatrix} = n$,
$\begin{pmatrix}
n\\n
\end{pmatrix} = 1$
\item $\begin{pmatrix}
n\\k
\end{pmatrix} = 0$ falls $k > n$\\
\item $0 \leq k \leq n \Rarr \begin{pmatrix}
n\\k
\end{pmatrix} = \frac{n!}{k! \cdot (n - k)!}$
\end{enumerate}
\Satz
$\begin{pmatrix}
n + 1\\k + 1
\end{pmatrix} = \begin{pmatrix}
n\\k
\end{pmatrix} + \begin{pmatrix}
n\\k +1
\end{pmatrix}$ für $0 \leq k \leq n$\\
Rekursionsformel\\
Beweis in der Übung.\\
Veranschaulichung: Das Pascal'sche Dreieck.
\Satz Anzahl der k-Elementigen Teilmengen einer n-elementigen Menge ist $\begin{pmatrix}
n\\k
\end{pmatrix}$.\\
Folgerung aus 2.9: $\begin{pmatrix}n\\k\end{pmatrix} \in \N_0$ für $1 ≤ k ≤ n$
\subsection*{Beweis zu 2.9:}
Ziehe k Kugeln aus einer Urne mit n nummerierten Kugeln, ohne gezogene Kugeln zurückzulegen.\\
(Zunächst unter Beachtung der Reihenfolge.)
\enum{
	\item Zug: n Möglichkeiten
	\item Zug: n-1 Möglichkeiten\\
	$\vdots$
	\item[k] Zug: n-k+1 Möglichkeiten
}
Nach Satz 2.6 kommt dabei jede k-elementige Teilmenge in $k!$ verschiedenen Anordnungen vor. (Reihenfolgen der Kugeln)\\
Anzahl der k-elementigen Teilmengen $\frac{n(n-1) \cdot ... \cdot (n-k+1)}{k!} = \begin{pmatrix}
n\\k
\end{pmatrix}$\\
\subsection*{Wichtige Anwendungen}
Seien $x, y\in \R, n \in \N_0$.\\
$(x+y)^n = ?$\\
$(x+y)^1 = x+y$
$(x+y)^2 = x^2 + 2xy + y^2$
$(x+y)^n = x^3 + 3x^2y + 3xy^2 + y^3$
Vermutung: Die Koeffizienten aus dem pascal'schen Dreieck sind Binominialkoeffizienten.\\