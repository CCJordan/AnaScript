% Kopfzeile beim Kapitelanfang:
\fancypagestyle{plain}{
%Kopfzeile links bzw. innen
\fancyhead[L]{\calligra\Large Vorlesung Nr. 9}
%Kopfzeile rechts bzw. außen
\fancyhead[R]{\calligra\Large 11.11.2013}
}
%Kopfzeile links bzw. innen
\fancyhead[L]{\calligra\Large Vorlesung Nr. 9}
%Kopfzeile rechts bzw. außen
\fancyhead[R]{\calligra\Large 11.11.2013}
% **************************************************
%
\section{Definition: Cauchy-Folge}
Eine Folge $(a_n)_{n\in \N}$ heißt Cauchy-Folge $\Leftrightarrow \forall \e > 0 \exists n_0\in \N : |a_n-a_m|<\e \forall n,m ≥ n_0$
\section{Satz}
Sei $(a_n)\subseteq \R$ konvergent $\Rarr$ $(a_n)$ ist Cauchy-Folge.\\
\subsection*{Beweis:}
$a_n \to a \Rarr |a_n - a_m| = |a_n - a + a -a_m| ≤ |a_n - a| + |a - a_m|$\\
Wähle $n_0$ so groß, dass $|a_n - a| < \frac{\e}{2} \forall n ≥ n_0$\\
$\Rarr \forall n, m ≥ n_0 : |a_n - a_m| < \frac{\e}{2} + \frac{\e}{2} = \e$\qed
\section{Satz: Cauchy-Kriterium}
Jede Cauchy-Folge $(a_n) \subseteq \R$ ist konvergent.\\
\subsection*{Beweis}
$(a_n)$ ist beschränkt, denn $\exists n_0 \in \N:|a_n - a_m|< 1 \forall n ≥ n_0$\\
$\Rarr |a_n| ≤ max\{|a_1|, ...,|a_{n_0}|,|a_{n_0}| + 1\} < \infty$\\
Bolzano-Weierstraß $\Rarr (a_n)$ hat konvergente Teilfolge $(a_{n_k})_{k\in\N}$\\
$a:=\lim\limits_{k\to\infty} a_{n_k}$\\
\ul{Behauptung}: $(a_n)_{n\in\N}$ konvergiert gegen $a$.\\
\subsection*{Beweis}
Sei $\e > 0 \Rarr |a_{n_k} - a| < \frac{\e}{2} \forall k ≥ N_1$\\
$|a_n - a_m| < \frac{\e}{2} \forall n, m ≥ N_2$\\
$k ≥ max\{N_1, N_2\} \Rarr |a_k - a| ≤ |a_k - a_{n_k}| + |a_{n_k} - a|$\\
$\Rarr |a_k - a| < \e$
\section{Definition: Uneigentliche Konvergenz}
$(a_n)_{n\in \N} \subseteq \R$ heißt \ul{uneigentlich Konvergent} gegen $+∞$ [gegen $-∞$] falls gilt:\\
$\forall M>0 \exists n\in \N:a_n>M \forall n ≥ n_0$\\
$[\forall M>0 \exists n\in \N:a_n<-M \forall n ≥ n_0]$\\
Schreibweise: $\lim\limits_{n\to ∞} a_n = +∞$
\section{Lemma:}
$\lim\limits_{n\to ∞} a_n = +∞ \Leftrightarrow \exists n_0 \in \N: a_n \neq 0 \forall n ≥ n_0$ und $\lim\limits_{n\to ∞} \frac{1}{a_n} = 0$\\
Denn: Für $M>0$ gilt:\\
$a_n > M \forall n ≥ n_0 \Leftrightarrow 0 < \frac{1}{a_n} < \frac{1}{M} \forall n ≥ n_0$
\section{Definition: Landau-Symbole}
Seien $(a_n), (b_n) \subseteq$ Folgen. Man schreibt:
\enum{
	\item $a_n = \mathcal{O}(b_n)$ für $n \to ∞$ falls $\exists C > 0$ und $n_0 \in \N: |a_n| ≤ C\cdot |b_n| \forall n ≥ n_0$\\
	Anschaulich: $(a_n)$ wächst höchstens so schnell wie $(b_n)$
	\item $a_n = \Theta (b_n)$ für $n\to ∞$, falls $a_n = \mathcal{O}(b_n)$ und $b_n = \mathcal{O}(a_n)$\\
	$a_n$ und $b_n$ wachsen gleich schnell.
	\item $a_n = \mathcal{o} (b_n)$ für $n\to ∞$ falls $\lim\limits_{n \to ∞} \frac{a_n}{b_n} = 0$\\
	Das heißt $b_n$ wächst schneller als $(a_n)$
}
\chapter{Komplexe Zahlen}
Motivation: Die Gleichung $x^2 + 1 = 0$ hat keine Lösung in $\R$ (denn $x\in \R \Rarr x^2 > 0 \Rarr x^2 + 1 > 0$)\\
\ul{Ziel}: Konstruktion eines Körpers $\C$, der $\R$ umfasst und in dem $x^2 + 1 = 0$ lösbar ist.\\
Vorüberlegung: Angenommen es gibt einen Körper $\C$ mit $\R \subseteq \C$ sodass $x^2 + 1 = 0$ eine Lösung $i \in \C$ hat.\\
Rechenregeln in einem Körper $\Rarr x, y, u, v \in \R$ gilt:\\
\enum {
	\item $(x + iy) + (u + iv) = (x + u) + i(y + v)$
	\item $(x + iy) \cdot (u + iv) = xu + i^2(yv) + i(xv + yu) = xu -yv + i(xv +yu)$
}
\section{Definition Körper der komplexen Zahlen}
Addition: $(x, y) + (u, v) = (x + u, y + v)$\\
Multiplikation: $(x, y) \cdot (u, v) = (xu -yv, xv + yu)$
\section{Satz: Die Komplexen Zahlen bilden einen Körper}
$(\C, +, \cdot)$ ist ein Körper (Körper der Komplexen Zahlen)
\subsection*{Beweis}
$+, \cdot$ sind kommutativ, + ist assoziativ, Distributiv-Gesetze sind erfüllt.
Neutrale Elemente:\\
Bezüglich Addition: 0 = (0, 0)\\
Bezüglich Multiplikation: 1 = (1, 0)\\
Inverse: -(x, y) = (-x, -y)\\
$(x, y) \neq 0 \Rarr (x, y)^{-1} = \left(\frac{x}{x^2 + y^2}, \frac{-y}{x^2 + y^2}\right)$\\
\section*{$\R$ als Teilkörper von $\C$}
$(x, 0) + (y, 0) = (x + y, 0)$ \\
Man identifiziert daher $x \in \R$ mit $(x, 0) \in \C$ und fasst $\R$ als Teilkörper von $\C$ auf.\\
Bezeichnung: $i := (0,1) \in \C$ nennen wir \ul{Imaginäre Einheit}\\
Damit $i^2 = (0,1) \cdot (0,1) = (-1, 0) = -1$\\
$\Rarr i $ und $-i = -1 \cdot i$ lösen die Gleichung $x^2 + 1 = 0$\\
Ferner: $\C \ni z = (x, y) \Leftrightarrow z = (x, 0) + (0, y)$\\
$\Leftrightarrow z = x + iy$\\
Jedes $z \in \C$ hat daher eine eindeutige Darstellung der Form:\\
$$z=x+iy \text{mit x, z \in $\R$}$$
\section{Definition}
Sei $z = x + iy \in \C$ mit $(x, y \in \R)$
$Re(z) := x$ \ul{Realteil} von $z$\\
$Im(z) := y$ \ul{Imaginärteil} von $z$\\
$z$ heißt reell, falls $Im(z) = 0$, $z$ heißt imaginär, falls $Re(z) = 0$
%TODO Geometrische Veranschaulichung
\section{Komplexe Konjugation und Betrag}
Sei $z = x + iy \in \C$ mit $x, y \in \R$\\
$\overline{z} = x - iy$ \ul{komlex Konjugierte Zahl}\\
Geometrisch: $z \mapsto \overline{z}$ ist die Spiegelung an der Reellen Achse.
\section{Eigenschaften der komplex konjugierten Zahl}
\enum{
	\item $\overline{z} = z$
	\item $\overline{z + w} = \overline{z} + \overline{w}$
	\item $z + \overline{z}$
	\item $z + \overline{z} = 2 Re(z)$, $z \cdot \overline{z} = 2 Im(z)$
	\item $z = \overline{z} $
	\item $z\overline{z} = x^2 +y^2 \in \R_+$
}