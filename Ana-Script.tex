\documentclass[a4paper,10pt]{scrreprt}
\usepackage{amsmath}	% Mathebibliothek 2
\usepackage{amssymb}	% Mathebibliothek 2
\usepackage{breqn}
\usepackage{wasysym}
\usepackage{fontspec}
\usepackage{polyglossia}
\usepackage[math-style=TeX]{unicode-math}	% Unicode Unterstützung
%\setmainfont{FreeSerif.otf}
%\setmathfont{Asana-Math.otf}
%\setmathfont{xits-math.otf}
\usepackage{calligra}	% Schrift
\usepackage{cancel}		% Kürzen und Durchstreichen
\usepackage{soulutf8}	% Für schöne Unterstreichungen
\usepackage[hyperindex,backref,hidelinks]{hyperref}

% Language
\setdefaultlanguage{german}

% Definitionen fürs floaten
\fboxsep .5pt
\fboxrule .5pt

% set underline
\setul{.3mm}{.8pt}

% Schusterjungen und Hurenkinder
\clubpenalty = 10000
\widowpenalty = 10000
\displaywidowpenalty = 1000

% Beschreibaren Seitenbereich definieren
\usepackage[left=2.5cm,right=2.5cm,top=2cm,bottom=2.5cm]{geometry}

% Kopf- und Fußzeile einfügen
\usepackage{fancyhdr}
\pagestyle{fancy}
\setlength{\headheight}{25pt}

% Kopfzeile Linie oben
\renewcommand{\headrulewidth}{0.5pt}

% Fußzeile mittig
\fancyfoot[C]{\thepage}

% Linie unten
\renewcommand{\footrulewidth}{0.5pt}

% Kopfzeilen fürs Inhaltsverzeichnis:
% Kopfzeile links bzw. innen
\fancyhead[L]{\calligra\Large Inhaltsverzeichnis}
% Kopfzeile rechts bzw. außen
\fancyhead[R]{\calligra\Large{}}

% ***********************
% *** costum commands ***
% ***********************

% kürzerer command um neue mathemathische commands (ohne parameter) zu erstellen
\newcommand{\mcmd}[2]{\newcommand{ #1 }{\text{$#2$}}} 
% kürzerer command um neue temxt commands (ohne parameter) zu erstellen
\newcommand{\tcmd}[2]{\newcommand{ #1 }{\text{#2}}} 
% kürzerer command um neue commands (ohne parameter, ohne environment) zu erstellen
\newcommand{\cmd}{\newcommand}

% Mengen
\mcmd{\N}{\mathbb{N}}
\mcmd{\Z}{\mathbb{Z}}
\mcmd{\Q}{\mathbb{Q}}
\mcmd{\R}{\mathbb{R}}
\mcmd{\K}{\mathbb{K}}
\renewcommand{\C}{\text{$\mathbb{C}$}}

\mcmd{\nN}{n\in\N}
\mcmd{\eN}{\in\N}
\mcmd{\eZ}{\in\Z}
\mcmd{\eQ}{\in\Q}
\mcmd{\eR}{\in\R}
\mcmd{\eC}{\in\C}

% griechische Buchstaben
\mcmd{\z}{\zeta}
\mcmd{\e}{\epsilon}	
\mcmd{\de}{\delta}	

% Pfeile
\mcmd{\equ}{\Leftrightarrow}
\mcmd{\Rarr}{\Rightarrow}
\mcmd{\Larr}{\Leftarrow}

% Symbole
\cmd{\qed}{\hfill\text{\calligra q.e.d.}}
\mcmd{\bs}{\backslash}
\mcmd{\ba}{\bs}

% häufige mathemathische Ausdrücke
\mcmd{\nif}{n→∞}
\mcmd{\an}{(a_n)}
\cmd{\til}[1]{\widetilde{#1}}

% Formelkürzel
\cmd{\bino}[2]{\text{$\ds\binom{#1}{#2} = \frac{#1!}{#2!\cdot(#1-#2)!}$}}

% Überschriften
\cmd{\sS}[1]{\section{#1}}
\cmd{\usS}[1]{\section*{#1}}
\cmd{\Def}{\sS{Definition:}}
\cmd{\Satz}{\sS{Satz:}}
\cmd{\uS}[1]{\section*{\ul{#1}}}
\cmd{\Wdh}[1]{\section*{Wiederholung #1}}
\cmd{\wdh}{\Wdh{}}
\cmd{\sss}[1]{\subsection*{\ul{#1}}}
\cmd{\bsp}{\sss{Beispiel:}}
\cmd{\Bsp}[1]{\subsection*{\ul{Beispiel:} #1}}
\cmd{\bem}{\sss{Bemerkung:}}
\cmd{\Bem}[1]{\subsection*{\ul{Bemerkung} #1:}}
\cmd{\beh}{\sss{Behauptung:}}
\cmd{\bew}{\sss{Beweis:}}
\cmd{\Bew}[1]{\subsection*{\ul{Beweis} #1:}}
\cmd{\anm}{\sss{Anmerkung:}}
\cmd{\ssss}[1]{{\bf\ul{#1}}}

% Textkürzel
\cmd{\ok}{\hfill{\checkmark}}

% Environments
\cmd{\ds}{\displaystyle}
\cmd{\Sum}{\ds\sum}
\cmd{\Lim}{\ds\lim}
\cmd{\Int}{\ds\int}

\cmd{\desc}[2]{\begin{description}\item[\textrm{#1}]{\hfill\\*#2}\end{description}}
\cmd{\ind}[2]{\desc{IA:}{$n=0$\\*#1\ok}\desc{IS:}{$n\to n+1$\\*#2\qed}}
\cmd{\indIV}[3]{\desc{IA:}{$n=0$\\*#1\ok}\desc{IV:}{#2}\desc{IS:}{$n\to n+1$\\*#3\qed}}
\cmd{\notat}[1]{\begin{description}\item[\textsf{\em Notation:}]{\hfill\\*\textsf{\em #1}}\end{description}}
\cmd{\ol}[1]{\text{$\overline{#1}$}}
\cmd{\ary}[1]{\begin{array}{c}#1\end{array}}
\cmd{\alg}[1]{\begin{align*}#1\end{align*}}
\cmd{\case}[1]{\begin{cases}#1\end{cases}}
\cmd{\enum}[1]{\begin{enumerate}#1\end{enumerate}}
\cmd{\itm}[1]{\begin{itemize}#1\end{itemize}}

% Author, title…
\title{Script zur Analysis Vorlesung 2013}
\author{Christopher Jordan}
\date{\today}

% Less detailed TOC
\setcounter{tocdepth}{1}

\begin{document}
\maketitle
\tableofcontents\newpage
% Kopfzeile beim Kapitelanfang:
\fancypagestyle{plain}{
%Kopfzeile links bzw. innen
\fancyhead[L]{\calligra\Large Vorlesung Nr. 1}
%Kopfzeile rechts bzw. außen
\fancyhead[R]{\calligra\Large 14.10.2013}
}
%Kopfzeile links bzw. innen
\fancyhead[L]{\calligra\Large Vorlesung Nr. 1}
%Kopfzeile rechts bzw. außen
\fancyhead[R]{\calligra\Large 14.10.2013}
% **************************************************
%
\chapter{Logik und Mengen}
\section*{Aussagen}
Eine Mathematische Aussage ist ein sprachlicher Ausdruck, dem genau einer der Wahrheitswerte "wahr" (w) oder "falsch" (f) zugeordnet werden kann. Eine Aussage im Mathematischen Sinne stellt eine objektive Tatsache dar.
\bsp
\textit{5 = 3 + 2} ist eine wahre Aussage.\\
\textit{2 $\cdot$ 3 = 9} ist eine falsche Aussage.\\
\textit{Das Essen ist gut gewürzt.} ist gar keine Aussage, da es sich hierbei nicht um ein objektive Faktum handelt.
\section{Verknüftung von Aussagen}
Liefert neue Aussagen. Seien $P, Q$ Aussagen.
\begin{enumerate}
\item \underline{Konjunktion}: $P \wedge Q$ ($P$ und $Q$)\\
Wahr genau dann, wenn P und Q wahr sind.
\item \underline{Disjunktion}: $P \vee Q$ ($P$ oder $Q$)\\
Wahr genau dann, wenn mindestens eine der Aussagen wahr ist.
\item \underline{Negation}: $\neg P$\\
Wahr g.d.w. $P$ falsch
\item \underline{Implikation}: $P \to Q$ (P impliziert Q, aus P folgt Q)\\
falsch g.d.w. P wahr und Q falsch, sonst wahr.\\
Bemerkung: Ist P falsch, so ist $P \to Q$ stets wahr.
\item \underline{Äquivalenz}: $P \leftrightarrow Q$ :=  $P \to Q \wedge Q \to P$ ($P$ äquivalent zu $Q$)\\
wahr g.d.w. $P$ und $Q$ den gleichen Wahrheitswert haben.
\end{enumerate}
% Wahrheitstabelle einfügen.
\section{Bezeichnung: 'semantische Äquivalenz'}
Zwei Aussagen $P, Q$, welche die gleiche Wahrheitstabelle haben, heißen semantisch äquivalent $P \equiv Q$\\
Bsp.
\begin{enumerate}
\item $\neg(\neg P) \equiv P$ denn % Wahrheitstabelle
\item De Morgan'sche Gesetze:\\
$\neg (P \wedge Q) \equiv \not P \vee \neg Q$\\
$\neg (P \vee Q) \equiv \not P \wedge \neg Q$\\
\item Distributivgesetze:\\
$P \wedge (Q \vee R) \equiv (P \wedge Q) \vee (P \wedge R)$\\
$P \vee (Q \wedge R) \equiv (P \vee Q) \wedge (P \vee R)$\\
\end{enumerate}
\subsection*{Mengen}
Def. nach G. Cantor Eine Menge ist eine Zusammenfassung bestimmter wohlunterschiedener Objekte unseres Denkens zu einem Ganzen. Diese Objekte heißen die Elemente der Menge.\\
Def. ist vage.\\
Pragmatischer Standpunkt: Eine Menge ist gebildet, wenn feststeht welche Elemente dazugehören.\\
\subsection*{Schreibweisen}
$x \in A$ heißt $x$ ist Element der Menge $A$\\
$x \notin A$ falls $x$ nicht Element der Menge $A$\\
$A \subseteq B$ (A teilmenge von B), falls jedes Element aus A auch in B liegt.\\
Entsprechend $B \supseteq A (B Obermenge von A)$\\
\begin{itemize}
\item $A = B$ falls $A \subseteq B$ und $B \supseteq A$
\item strikte Inklusion $A \subset B$ A ist Teilmenge von $B$ g.d.w. ($A \neq B$)
\end{itemize}

\subsection{Beschreibung von Mengen}
\begin{enumerate}
\item Durch Aufzählung der Elemente\\
Bsp.: $A = \{Rot, Blau\}$, $A = \{1, \{1, 2\}, \{1, 2,\{1, 2\} \}\}$\\
$\mathbb{N} = \{1, 2, 3\}$ Menge der natürlichen Zahlen\\
$\mathbb{N}_0 = \{0, 1, 2, 3\}$\\
$\mathbb{Z} = \{0, \pm 1, \pm 2\}$
\item durch eine charakteristische Eigenschaft\\
$A = \{n \in \mathbb{Z}: N ist gerade\} = \{n \in \mathbb{Z}: \text{es gibt} k \in Z \text{ mit } n = 2k\}$
\end{enumerate}
\section{Definition Primzahlen}
Seien $n, m \in \mathbb{Z}$. m heißt Teiler von n, kurz $m | n$ falls es ein $k \in \mathbb{Z}$ gibt mit $n = k \cdot m$. Sonst $m \not | n$.
\section{Mengenoperationen}
Seien $A, B$ Mengen.
\begin{description}
\item[Vereinigung der Mengen $A$ und $B$]{$A \cup B := \{x: x \in A \vee x\in B\}$\\
Die Vereinigung aus $A$ und $B$ besteht aus den Elementen die in $A$ oder $B$ enthalten sind.}
\item[Durchschnitt der Mengen $A$ und $B$] {$A \cap B := \{x: x \in A \wedge x \in B\}$\\
Der Durchschnitt von $A$ und $B$ besteht aus den Elementen, die sowohl in $A$ als auch in $B$ enthalten sind.}
\item[Differenzmenge]{$A \setminus B := \{x: x \in A \wedge x \notin B\}$\\
Die Differenzmenge von $A$ und $B$ besteht aus jenen Elementen, die in $A$ jedoch nicht in $B$ enthalten sind.}
\end{description}
\section{Distributivgesetze}
\begin{enumerate}
\item $A \cap (B \cup C) = (A \cap B) \cup (A \cap C)$
\item $A \cup (B \cap C) = (A \cup B) \cap (A \cup C)$
%Beweis
\end{enumerate}
\section{Kartesisches Produkt von Mengen}
Seien $A, B$ Mengen.\\
$A \times B := {(a, b): a \in A, b\in B}$ Mengen aller geordneten Paare $(a,b)$ im Gegensatz zu $\{a, b\} = \{b,a\}$ kommt es bei $(a, b)$ auf die Reihenfolge an.
\section{Bezeichnung}
Sei A eine Menge.
\begin{enumerate}
\item $\mathcal{P}(A) =\{B: \subseteq A\}$ Potenzmenge von $A$
\item Ist $A$ endlich, so bezeichnet $\vert A \vert$ die Anzahl der Elemente von $A$ (Mächtigkeit von $A$)\\
Falls A unendlich schreibe (etwas lax) $\vert A \vert = \infty$
\end{enumerate}
% Kopfzeile beim Kapitelanfang:
\fancypagestyle{plain}{
%Kopfzeile links bzw. innen
\fancyhead[L]{\calligra\Large Vorlesung Nr. 2}
%Kopfzeile rechts bzw. außen
\fancyhead[R]{\calligra\Large 17.10.2013}
}
%Kopfzeile links bzw. innen
\fancyhead[L]{\calligra\Large Vorlesung Nr. 2}
%Kopfzeile rechts bzw. außen
\fancyhead[R]{\calligra\Large 17.10.2013}
% **************************************************
\chapter{Natürliche Zahlen und vollständige Induktion}
Vollständige Induktion ist ein Beweisprinzip für Aussagen über natürliche Zahlen.\\
Es beruht auf dem:
\section{Induktionsaxiom}
Sei $M \subseteq \N$ mit
\enum{
\item $1 \in M$
\item Für alle $n \in \N$ gilt: $n \in M \Rarr n + 1 \in M$
}
Dann ist $M = \N$\\
\bem Axiome sind Grundlegende Aussagen in einer Theorie, die ohne Beweis vorausgesetzt werden.
\section{Satz: Prizip der vollständigen Induktion}
Zu jedem $n \in \N$ sei eine Aussage $A(n)$ gegeben. Es gelte
\begin{enumerate}
\item {A(1) ist wahr ( {Induktionsanfang})}
\item Für alle $n \in \N$ gilt: $A(n) \Rarr A(n+1)$
\end{enumerate}
Dann ist $A(n)$ wahr für alle $n \in \N$\\
"Dominoeffekt" $A(1)$ ist wahr $\Rarr A(2)$ ist wahr.\\
M erfüllt (1) und (2) das Induktionsaxiom. Also ist $M = \N$\\
\\
%
Verschiebung des Induktionsanfangs:\
Seien A(n) Aussage für $n \in \Z, n \geq n_0$ ($n_0 \in \Z$ fest)\\
Dann gilt das Prinzip der vollständigen Induktion entsprechend mit Induktionsanfang bei $n_0$ statt bei 1.
\section*{Summen und Produktzeichen}
Gegeben seien reelle Zahlen $a_n$ für $m \leq k \leq n$ wobei $m,n \in \Z, m \leq n$ Man setzt 
$$\sum_{k=m}^{n} a_k := a_m + a_{m+1} + ... + a_{n} $$
und 
$$ \prod_{k = m}^{n} a_k := a_m \cdot a_{m+1} \cdot ... \cdot a_n $$
Konvention für $m > n$:\\
$\sum_{k=m}^{n} a_k = 0$
und
$\prod_{k=m}^{n} a_k = 1$
\subsubsection*{Beispiel: Die Arithmetische Reihe}
$\sum_{k = 1}^{n} = 1 + 2 + ... n = ?$\\
Die Idee von Gauß für n = 100\\
$1 + ... 100 = (1 + 100) + (2 + 99) + ... (50 + 51) = 101 \cdot 50 = 5050$\\
\section{Satz}
$\sum_{k=1}^{n} k = \frac{1}{2} \cdot n \cdot (n+1)$ für alle $n \in \N$\\
Beweis mit vollständiger Induktion:\\
$A(n) = \sum_{k=1}^{n} k = \frac{1}{2} \cdot n \cdot (n + 1)$
% römische Zahlen
\begin{enumerate}
\item Induktionsanfang $n = 1$\\
A(1) ist wahr, denn $\sum_{k=1}^{1} k = \frac{1}{2} \cdot 1 \cdot 2$ \ok
\item Induktionsschluss $n \to n + 1$\\
Es sei A(n) wahr, das heißt, es gelte:\\
$\sum_{k = 1}^{n} = \frac{n (n+1)}{2}$ (Induktionsvoraussetzung)\\
$\Rarr \sum_{k = 1}^{n + 1} k = \sum_{k = 1}^{n} + (n + 1) \underset{I.V.}{=} \frac{1)}{2} \cdot n \cdot (n+1) + (n+1) = (n+1)(\frac{n}{2} + 1) = \frac{1}{2} (n+2) \cdot (n+1)$
\Rarr A(n+1) ist wahr. Mit vollständiger Induktion folgt die Behauptung.
\end{enumerate}
\bem zum Rechnen mit Summen (analog für Produkte):
$\sum_{K = 1}^{n + 1} a_k = \sum_{k = 1}^{n} a_k + a_{n+1}$\\
für $m \in \N$: $\sum_{k=1}^{m+n} a_k = \sum_{k = 1}^{n} a_k + \sum_{k = n + 1}^{m} a_k$
\subsection*{Beispiel 2: Geometrische Reihe}
Potenzen: Sei $x\in \R, n \in \N_0$ dann $x^n := x_1 \cdot x_2 \cdot ... \cdot x_n = \prod_{k=1}^{n} x$
Indexsumme $x^0 = 1$
\section{Satz: Gemetrische Summenformel}
Für $x \in \R\setminus \{1\}$ und $n \in \N_0$ gilt:
$$\sum_{k = 0}^{n} x^k = \frac{1 - x^{n+1}}{1 - x}$$
\subsubsection*{Beweis mit vollständiger Induktion}
% Römische Zahlen
\begin{enumerate}
\item Ind. Anfang $n = 0$: $x^0 = \frac{1 - x}{1 - x} = 1$
\item Ind. Schluss $n \to n + 1$\\
I.V.: $\sum_{k = 0}^{n} x^k = \frac{1 - x^{n+1}}{1 - x}$\\
\Rarr $\sum_{k = 0}^{n + 1} x^k = \sum_{k = 0}^{n} x^k + x^{n+1} = \frac{1 - x^{n+1}}{1 - x} + x^{n+1} 
= \frac{1 - x^{n+1} + (1-x)\cdot x^{n+1}}{1 - x} = \frac{1 - x^{n+2}}{1 - x}$
\end{enumerate}
\qed\\
Ein Beweis mit vollständiger Induktion setzt voraus, dass man die zu beweisende Identität bereits kennt, bzw. sie vermutet.\\
Eine solche Vermutung gewinnt man z.B. durch Berechnungen für kleine Werte von n.
\section{Definition: Fakultät}
Für $n \in \N_0$ setze $n! := \prod_{k=1}^{n} k = 1 \cdot 2 \cdot ... \cdot n$\\
Also $0! = 1$, $1! = 1$, $2! = 3$, $3! = 6$\\
$n! = (n+1)! \cdot n$\\
$n!$ wächst sehr schnell. z.B. $13! \approx 6,2 \times 10^9$\\
Es gibt dafür eine einfache Formel analog zur arithmetischen Summenformel.\\
\Satz
Die Anzahl der möglichen Anordnungen (auch Permutationen) der Elemente einer Menge M mit $|M| = n$ ist $n!$.
Beweis (bezeichne die Elemente 1, 2, ... n):
\enum{
\item Induktionsanfang: n = 1\\
1 = 1! eine Mögliche Anordnung. \ok
\item Induktionsschluss $n \to n + 1$\\
Besetze zunächst Position 1 dafür gibt es $n + 1$ Möglichkeiten.\\
Sei $P_k := \{$Permutationen von $1, 2, ... n + 1$ mit $k$ an Position 1\}\\
Nach I.V. ist $|P_k| = n!$ (Anzahl der Möglichkeiten die Stellen 2, 3, ... n+1 zu besetzen)
\Rarr Anzahl der Permutationen von $1, 2... (n+1) = (n+1)!$
}
\section{Definition: Binomialkoeffizienten}
Seinen $k, n \in \N_0$ $\begin{pmatrix}
n\\k
\end{pmatrix} := \prod_{j = 1}^{k} \frac{(n - j + 1)}{j}$\\
sprich "k aus n" oder "n über k"\\
Folgerungen:
\begin{enumerate}
\item $\begin{pmatrix}
n \\ 0
\end{pmatrix} = 1$,
$\begin{pmatrix}
n\\1
\end{pmatrix} = n$,
$\begin{pmatrix}
n\\n
\end{pmatrix} = 1$
\item $\begin{pmatrix}
n\\k
\end{pmatrix} = 0$ falls $k > n$\\
\item $0 \leq k \leq n \Rarr \begin{pmatrix}
n\\k
\end{pmatrix} = \frac{n!}{k! \cdot (n - k)!}$
\end{enumerate}
\Satz
$\begin{pmatrix}
n + 1\\k + 1
\end{pmatrix} = \begin{pmatrix}
n\\k
\end{pmatrix} + \begin{pmatrix}
n\\k +1
\end{pmatrix}$ für $0 \leq k \leq n$\\
Rekursionsformel\\
Beweis in der Übung.\\
Veranschaulichung: Das Pascal'sche Dreieck.
\Satz Anzahl der k-Elementigen Teilmengen einer n-elementigen Menge ist $\begin{pmatrix}
n\\k
\end{pmatrix}$.\\
Folgerung aus 2.9: $\begin{pmatrix}n\\k\end{pmatrix} \in \N_0$ für $1 ≤ k ≤ n$
\subsection*{Beweis zu 2.9:}
Ziehe k Kugeln aus einer Urne mit n nummerierten Kugeln, ohne gezogene Kugeln zurückzulegen.\\
(Zunächst unter Beachtung der Reihenfolge.)
\enum{
	\item Zug: n Möglichkeiten
	\item Zug: n-1 Möglichkeiten\\
	$\vdots$
	\item[k] Zug: n-k+1 Möglichkeiten
}
Nach Satz 2.6 kommt dabei jede k-elementige Teilmenge in $k!$ verschiedenen Anordnungen vor. (Reihenfolgen der Kugeln)\\
Anzahl der k-elementigen Teilmengen $\frac{n(n-1) \cdot ... \cdot (n-k+1)}{k!} = \begin{pmatrix}
n\\k
\end{pmatrix}$\\
\subsection*{Wichtige Anwendungen}
Seien $x, y\in \R, n \in \N_0$.\\
$(x+y)^n = ?$\\
$(x+y)^1 = x+y$
$(x+y)^2 = x^2 + 2xy + y^2$
$(x+y)^n = x^3 + 3x^2y + 3xy^2 + y^3$
Vermutung: Die Koeffizienten aus dem pascal'schen Dreieck sind Binominialkoeffizienten.\\
% Kopfzeile beim Kapitelanfang:
\fancypagestyle{plain}{
%Kopfzeile links bzw. innen
\fancyhead[L]{\calligra\Large Vorlesung Nr. 3}
%Kopfzeile rechts bzw. außen
\fancyhead[R]{\calligra\Large 21.10.2012}
}
%Kopfzeile links bzw. innen
\fancyhead[L]{\calligra\Large Vorlesung Nr. 3}
%Kopfzeile rechts bzw. außen
\fancyhead[R]{\calligra\Large 21.10.2013}
% **************************************************
%
\section{Der Binomische Satz}
Seien $x, y \in \R, n \in \N_0$
$$(x+y)^n = \sum_{k=0}^{n} \begin{pmatrix}n\\k\end{pmatrix} x^k y ^{n-k}$$
\subsection*{Beweis}
Mit Induktion nach $n$:
\enum{
	\item[I] Induktionsanfang $n=0$:\\
	$(x + y)^0 = 1 = \begin{pmatrix}0\\0\end{pmatrix} x^0y^0$
	\item[II] Induktionsschritt $n\to n + 1$:\\
	$(x+y)^{n+1} = (x+y)^1 + (x+y)^n \underset{I.V.}{=} (x+y)^1 \cdot \left( \sum_{k=0}^{n} \begin{pmatrix}n\\k\end{pmatrix} x^k y ^{n-k} \right)$\\
	$\Rarr x\cdot (x) + y \cdot (x)$\\
	$\Rarr \sum_{k=0}^{n} \begin{pmatrix}n\\k\end{pmatrix} x^{(k + 1)} y ^{n-k} + \sum_{k=0}^{n} \begin{pmatrix}n\\k\end{pmatrix} x^k y ^{n-k + 1}$\\
	$\underset{Indexverschiebung}{\Rarr}\sum_{k=1}^{n+1} \begin{pmatrix}n\\k\end{pmatrix} x^k y ^{n-(k-1)}$
	$\Rarr (x+y)^{n+1} = y^{n+1} + x^{n+1} + \sum_{k=1}^{n} \left[\begin{pmatrix}n\\k\end{pmatrix} + \begin{pmatrix}n\\k-1\end{pmatrix}\right] x^k y ^{n+1-k}$\\
	$\Rarr \sum_{k=0}^{n+1} \begin{pmatrix}n+1\\k\end{pmatrix} x^k y ^{n-(k-1)}$ \qed
}
\chapter{Die Reellen Zahlen}
Ein Ziel bei der Erweiterung von Zahlenbereichen ist die Lösbarkeit von Gleichungen.\\
$\N \to \Z: x + n = m \qquad (n,m \in \N_0)$\\
$\Z \to \Q: x \cdot n = m \qquad (n,m \in \Z, n \neq 0)$\\
Aber $x^n = m (n,m \in \N)$ hat im Allgemeinen keine Lösungen im $\Q$
\section*{Axiomatische Einführung von $\R$}
Wir geben eine Reihe von Grundlegenden Eigenschaften von $\R$ an.
\enum{
	\item Körperaxiome
	\item Anordnungsaxiome
	\item Vollständigkeitsaxiom
}
Man kann zeigen (wichtiger Satz): Es gibt genau 1 Menge $\R$ mit diesen Eigenschaften.\\
Es gibt präzise Konstruktionen\\
$\N \to \Z \to \Q \underset{\text{Vervollständigung}}{\longrightarrow} \R$ so dass $\N \subseteq \Z \subseteq \Q \subseteq \R$
\section{Vorbemerkung: Quantoren}
Gegeben: Aussagen $P(x), x\in X (X \text{ sei Menge})$\\
\begin{tabular}{l|l}
Aussage & Schreibweise\\\hline
Für alle $x\in X$ gilt $P(X)$ & $\forall x\in X: P(x)$\\
Es gibt (mindestens) ein x mit $P(x)$ & $\exists x \in X: P(x)$\\
Es gibt genau ein $x \in X$ mit $P(x)$ & $\exists! x \in X: P(x)$\\
Es gibt kein $x \in X$ mit $P(x)$ & $\not\exists x\in X: P(x)$
\end{tabular}\\
$\forall: Allquantor, \exists: Existenzquantor$
\section*{I. Körperaxiome}
Auf der Menge $\R$ sind zwei Rechenoperationen + (Addition) und $\cdot$ (Multiplikation) erklärt, sodass $(\R, +, \cdot)$ ein Körper ist.
\section{Definition: Körper}
Ein Körper ist eine Menge $\mathbb{K}$ mit zwei Operationen + (Addition) und $\cdot$ Multiplikation, sodass gilt:
\enum{
	\item $\forall x, y \in \mathbb{K}$:\\
	$(x+y) + z = x +(y+z)$ (Assoziativität für +)\\
	$(x\cdot y) \cdot z = x \cdot (y \cdot z)$ (Assoziativität für $\cdot$)
	\item $\forall x,y \in \mathbb{K}$:\\
	$x+y = y + x$ (Kommutativität für +)\\
	$x \cdot y = y \cdot x$ (Kommutativität für $\cdot$)\\
	\item Existenz neutraler Elemente:\\
	$\exists 0 \in \mathbb{K}$ (Null) mit: $x + 0 = x$\\
	$\exists 1 \in \mathbb{K}$ (Eins) mit: $x \cdot 1 = x$
	\item Existenz von Inversen:\\
	$\forall x \in \mathbb{K} \exists y \in \mathbb{K} x + y = 0$ (additives Inverses von x)\\
	$\forall x \in \mathbb{K}\setminus \{0\} \exists z \in \mathbb{K} x + z = 1$ (multiplikatives Inverses von x)
	\item Distributivgesetz\\
	$\forall x,y,z \in \mathbb{K}: x \cdot (y + z) = x y + x z$
}
\section{Folgerungen}
Sei $K$ ein Körper.
\enum{
\item 0 und 1 sind eindeutig bestimmt.\\
Beweis für 0 (für 1 analog):\\
Sei 0' neutral bezüglich +, sodass $0 = 0 + 0' = 0' + 0 = 0'$
\item $y, z$ in (4) sind (bei festem x) eindeutig bestimmt:\\
Beweis für y: \\
Sei $x + y = 0 = x + y $\\
$\Rarr y = y + 0 = y + (x + y') = (y + x) + y' = (x + y) + y' = y'$\\
Bez. $-x$: additives Inverses\\
$x^{-1} = \frac{1}{x}$
\item $a, b\in \mathbb{K} \Rarr$ die Gleichung $a + x = b$ hat eine ein eindeutige Lösung, nämlich x = b+ (-a)
}
% Kopfzeile beim Kapitelanfang:
\fancypagestyle{plain}{
%Kopfzeile links bzw. innen
\fancyhead[L]{\calligra\Large Vorlesung Nr. 4}
%Kopfzeile rechts bzw. außen
\fancyhead[R]{\calligra\Large 24.10.2013}
}
%Kopfzeile links bzw. innen
\fancyhead[L]{\calligra\Large Vorlesung Nr. 4}
%Kopfzeile rechts bzw. außen
\fancyhead[R]{\calligra\Large 24.10.2013}
% **************************************************
%
\subsection{Folgerungen}
\enum{
\item $\forall x \in \K : x \cdot 0 = 0$
\item $x \cdot y = 0 \Leftrightarrow x = 0 \vee y = 0$
\item $(-x) \cdot x = -x\cdot y$ insbesondere $(-1) \cdot y = -y$
}
\subsection*{Allgemeine A und K-Gesetze}
$x_1 + \cdots + x_n := (...((x_1 + x_2) + x_3) + ...)$
Wiederholte Anwendung des A- und K-Gesetzes zeigt: Das Ergebnis ist unabhängig von Klammerung ($\to$ Klammern weglassen) und Reihenfolge.\\
Ebenso für Produkte.\\
\subsubsection*{Potenzen}
$x\in \K$ Für $n \in \N$ setze $x^n := \underbrace{x\cdot ... \cdot x}_{\text{n Faktoren}}$, $x^0 = 1$\\
Falls $x \neq 0$: $x^{-n} := (x^{-1})^n$\\
Damit $x^{-n} = (x^{-1})^n$\\
Regeln $x,y \in \K, n, m \in \N_0$
$x^n\cdot x^m = x^{m+n}, x^n\cdot y^n = (x\cdot y)^n, (x^n)^m = x^{m\cdot n}$\\
Falls $x,y \neq 0$ gelten diese Regeln auch für alle $n, m \in \Z$
\section*{Beispiele}
$\R$ ist Körper, ebenso $\Q$ mit den üblichen Operationen.\\
$\Z$ nicht, da nur $+1$ und $-1$ multiplikative Inverse haben.\\
Beispiel eines Endlichen Körpers:
$\mathcal{F}_2 = \{0,1\}$ mit Operationen wie folgt:\\
\begin{tabular}{c|c|c}
+ & 0 & 1\\\hline
0 & 0 & 1\\
1 & 1 & 0
\end{tabular}
\begin{tabular}{c|c|c}
$\cdot$ & 0 & 1\\\hline
0 & 0 & 0\\
1 & 0 & 1
\end{tabular}
\section*{II. Anordnungsaxiome}
$\R$ enthält eine Teilmenge von Elementen, die als positiv ausgezeichnet sind. Wobei gelten:\\
\enum {
\item[A1] Jedes $x \in \R$ genügt genau einer der Beziehungen $x > 0$, $x = 0$ oder $x<0$.
\item[A2] $x>0 \wedge y > 0 \Rarr x + y > 0, x \cdot y > 0$
}
\section{Folgerungen}
\enum {
\item $x< 0 \Leftrightarrow -x > 0$
\item $\forall x,y \in \R$ gilt genau eine der Relationen $x > y, x = y, x < y$
\item $x < y \wedge y < z \Rarr x < z$ (Transitivität)
\item $x < y, z \in \R \Rarr x + z < y + z$
\item $x < y \Leftrightarrow -x < -x$
\item $x < y \wedge a < b \Rarr x + a < y + b$
\item $x < y \wedge a > 0 \Rarr a\cdot x < a \cdot y$\\
      $x < y \wedge a < 0 \Rarr a\cdot x > a \cdot y$
\item $0 ≤ x < y, 0 ≤ a < b \Rarr a \cdot x < b \cdot y$
\item $x \neq 0 \Rarr x^2 > 0$
}
\subsubsection*{Beweise}
\enum {
\item $x < 0 \overset{Def.}{\Leftrightarrow} 0 > x \overset{Def.}{\Leftrightarrow} 0-x > 0$
\item aus Folgerung 3
\item $z-x = (z-y) + (x-y)) >0$
\item Bilde Differenz
\item Bilde Differenz
\item $(y + b) - (x + a) = (y - x) + (b - a) > 0$
\item $a\cdot y - a\cdot x = a(x-y) > 0$ Gegenrichtung analog
\item Übung
\item Falls $x > 0 \Rarr x^2 > 0$\\
Falls $x < 0 \Rarr [7) mit y = 0, a = x] x^2 > 0$
}
\section{Definition}
Ein Körper $\K$ in dem eine Teilmenge von Elementen als positiv ausgezeichnet ist, i. Z. x > 0, sodass (A1) + (A2) gelten, heißt angeordneter Körper.\\
Folgerungen 3.4 gelten in jedem angeordneten Körper!
$\R, \Q sind angeordnete Körper$\\
$\mathcal{F}_2$ nicht. im $\mathcal{F}_2$ gilt $1 + 1 = 0$\\
Annahme: $\mathcal{F}_2$ angeordnet $\Rarr 1 > 0 \Rarr 1 + 1 > 0$\\
\section{Bernoulli Ungleichung}
Sei $x \in \R, x > -1$ und $n \in \N_0 \Rarr $ $$(1+x)^n ≥ 1 + n \cdot x$$
\subsection*{Beweis}
Induktion nach $n$\\
Induktionsanfang: $n = 0:\\
(1+x)^0 = 1 = 1 + 0 \cdot x$\\
Induktionsschritt: $n\to n+1:\\
(1+x)^{n+1} = (1 + x) \cdot (1+x)^n ≥ (1+x) \cdot (1+ n\cdot x) \\
= 1 + (n+1) \cdot x + n \cdot x^2 ≥ 1 + (n + 1) \cdot x$ \qed
\section{Definition: Betrag}
Sei $x\in \R$.\\
$\left\|x\right\| = \left\langle\begin{array}{l l}
x &  falls x ≥ 0\\
-x  &  falls x < 0
\end{array} \right.$
\section{Satz}
\enum {
\item $\left\|x\right\| ≥ 0$ wobei $\left\|x\right\| \Leftrightarrow x = 0$
\item $\left\|x \cdot y\right\| = \left\|x\right\| \cdot \left\|y\right\|$
\item $\left\|\frac{x}{y}\right\| = \frac{\left\|x\right\|}{\left\|y\right\|}$ falls $y \neq 0$
\item $\left\|x + y\right\| ≤ \left\|x\right\| + \left\|y\right\|$ (Dreiecksungleichung)
}
\subsection*{Beweise}
\enum {
\item klar.
\item $|xy| \in \{ \pm xy\}, |x| \cdot |x| \in \{ \pm x \cdot y \}$
\item $x = \frac{x}{y} \cdot y \Rarr \left|x\right| \cdot \left|\frac{x}{y}\right| \cdot \left|y\right| \Rarr$ Beh. 
\item $\pm x ≤ \left|x\right|, \pm y ≤ \left|y\right| \Rarr \pm (x+y) ≤ \left|x\right| + \left|y\right| \Rarr$ Beh.
}
\section*{Intervalle}
Seien $a, b \in \R$, $a ≤ b$\\
$\begin{array}{l l}
[a, b] &:= \{x \in \R: a ≤ x ≤ b\}\\
(a, b) &:= \{x \in \R: a< x < b \}\\
\left[ a, b\right) &:= \{x \in \R: a ≤ x < b\}\\
\left(a, b\right] &:= \{x \in \R: a < x ≤ b\}\\
\left[ a, \infty\right) &:= \{x \in \R: x ≥ a\}\\
\left(-\infty, a\right] &:= \{x \in \R: x ≤ a\}\\
\end{array}$
\section*{III. Vollständigkeitsaxiom}
\section{Definition Beschränkte Mengen}
$M \subseteq \R$ heißt nach oben [unten] beschränkt $ \Leftrightarrow $ Wenn es ein $s$ aus $\R$ gibt, sodass alle Zahlen aus M kleiner [größer] sind als $M$.\\
$s$ heißt dann obere [untere] Schranke von M.\\
Liegt die obere [untere] Schranke in $m \in M$, so nennt man sie Maximum [Minimun] von $M$.\\
$M$ heißt beschränkt $\Rarr$ M ist nach oben und unten beschränkt.
\subsection{Beispiel}
\enum {
\item $M = [0,1] \Rarr M$ ist beschränkt. Jedes $s ≥ 1$ ist eine obere Schranke, jedes $t ≤ 0$ ist eine untere Schranke von $M$.\\
1 ist Maximum von $M$, 0 ist Minimum von $M$.
\begin{description}
\item[\textit{m = max(M)}] schreiben wir für das Maximum von $M$.
\item[\textit{m = min(M)}] schreiben wir für das Minumum von $M$.
\end{description}
\item $M = [0,1)$ $M$ hat kein Maximum.\\
Kein $m \in M$ ist obere Schranke von $M$
da $m < \frac{1}{2} (m+1) < 1 \forall m \in M$
}
\desc{Beachte}{$M$ hat höchstens ein Maximum / Minimum!\\
Denn: Seien $m\neq m'$ Maxima von $M$, etwa $m < m'$ $\Rarr$ $m$ ist keine obere Schranke von $M$.}
% Kopfzeile beim Kapitelanfang:
\fancypagestyle{plain}{
%Kopfzeile links bzw. innen
\fancyhead[L]{\calligra\Large Vorlesung Nr. 5}
%Kopfzeile rechts bzw. außen
\fancyhead[R]{\calligra\Large 28.10.2013}
}
%Kopfzeile links bzw. innen
\fancyhead[L]{\calligra\Large Vorlesung Nr. 5}
%Kopfzeile rechts bzw. außen
\fancyhead[R]{\calligra\Large 28.10.2013}
% **************************************************
%
\section{Definition: Supremum, Infimum}
\enum{
	\item $s\in \R$ heißt \ul{Supremum} von $M$ $\Leftrightarrow$ $s$ ist kleinste obere Schranke von $M$. \\
	Das heißt:
	\enum{
		\item $s$ ist obere Schranke von $M$
		\item $s≤s'$ für jede weitere obere Schranke $s'$ von $M$ 
	}
	\item $t \in \R$ heißt \ul{Infimum} von $M$ $\Leftrightarrow$ $t$ ist größte untere Schranke von $M$.
}
1.1 zeigt: $M$ hat höchstens ein Supremum $s$: Bezeichnung $s = sup(M)$\\
Denn: $s, s'$ seien Suprema $\Rarr$ $s≤s' \wedge s'≤s \Rarr s = s'$\\
Analog $M$ hat höchstens ein Infimum. Bezeichnung: $t = inf(M)$
\section{Vollständigkeitsaxiom (Supremumeigenschaft von $\R$)}
Sei $M \subseteq \R, M \neq \emptyset$ nach oben beschränkt $\Rarr M$ besitzt ein Supremum.
\section{Folgerung}
Sei $M \subseteq \R, M \neq \emptyset$  nach unten beschränkt, so besitzt $M$ ein Infimum. 
\desc{Beachte}{Hat $M$ ein Maximum $m = max(M)$ so ist zugleich $m = sup(M)$}
\section{Lemma}
Sei $M \subseteq \R, M \neq \R$ nach oben beschränkt, $s = sup(M)$\\
$\forall \e > 0 \exists x \in M: s - \e < x ≤ s$
\subsubsection*{Beweis}
$s-\e$ ist keine obere Schranke von M, da s kleinste obere Schranke. $\Rarr \exists x\in M: s-\e < x$\\
Dabei $x ≤ s$ da $s$ obere Schranke von M. $\qed$
\section{Satz: Archimedische Eigenschaft}
(AR) $\forall x \in \R \exists \underbrace{ n \in \N}_{\text{Abhängig von} x}: n > x$
\subsubsection*{Beweis}
Angenommen (AR) gilt nicht, d.h. $\exists x \in \R: x ≤ n$ für alle $n \in \N$\\
$\Rarr \N$ ist durch x nach oben beschränkt.\\
$s = sup(M)$ existiert.\\
$\underset{\Rarr}{Lemma 3.13} \exists n \in \N: s-1 < n $\\
Widerspruch gegen Supremum Definition.
\section{Folgerungen aus der Archimedischen Eigenschaft}
\enum {
\item $\forall \e > 0 \exists n \in \N$: $\frac{1}{n} < \e$
\item Wachstum von Potenzen\\
Sei $a \in \R, a > 1 \Rarr \forall M > 0 \exists n \in \N: a^n >M$
\item Sei $a\in \R, 0 < a < 1 \Rarr \forall \e > 0\exists n \in \N:a^n < \e$
}
\subsubsection*{Beweise}
\enum {
\item Sei $\e > 0$ (AR) $\to \exists n \in \N: n > \frac{1}{\e} \Rarr \frac{1}{n} < \e$
\item Sei $M > 0$ $x:= a-1 > 0 \Rarr a^n = (1+x)^n ≥ 1 + x \cdot n$\\
(AR) $\exists n \in \N: n > \frac{M}{x} \Rarr a^n > M$\\
\item Sei $\e > 0. \frac{1}{a}>1 \Rarr \exists n\in \N: \left(\frac{1}{a}\right)^n>\frac{1}{\e} \to a^n < \e \qed$
Man kann zeigen: Es gibt (bis auf Umbenennungen) genau einen angeordneten Körper, der das Vollständigkeitsaxiom erfüllt, nämlich $\R$.
}
$\forall \e > 0 \exists \nN \frac{1}{n} < \e$
\section{Satz: Existenz von Wurzeln}
Sei $a\in \R, a ≥ 0$ und $k \in \N\Rarr \exists! x \in \R, x≥0: x^k = a$\\
Schreibweise: $x= a^{\frac{1}{k}} = \sqrt[k]{a}$ $k$-te Wurzel aus $a$\\
$\sqrt{a} := \sqrt[2]{a}$\\
\desc{Beachte}{$\sqrt[k]{0} = 0$ für $a > 0$ ist $\sqrt[k]{a} > 0$(per Definition)}
\subsubsection*{Beweis}
\enum{
\item Eindeutigkeit:\\
Seien $x_1, x_2 ≥ 0 x_1 \neq x_2$ mit $x_1^k = a = x_2^k$\\
Sei etwa $x_1 < x_2 \underset{\Rarr}{3.4.8} x_1 < x_2$
\item Existenz:\\
Hier für $k = 2$ ($k > 2$ ähnlich, aber aufwendiger)\\
Sei o.E. (ohne Einschränkungen) $a ≥ 1$\\
Setze $M:=\{y>0: y^2 ≤ a\}$\\
$M \neq \emptyset$ (da $1 \in M$). $M$ ist nach oben durch $a$ beschränkt.\\
Angenommen es gebe $y \in M: y > a \Rarr y^2 > a^2 \underset{≥}{a≥1} a$\\  % TODO BLITZ!!
Setze $x:= suo(M)$. Behauptung: $x^2 = a$\\
Annahme 1:\\
$x^2 > a \Rarr \e := \frac{x^2 -a}{2x} > 0$\\
Lemma 3.13 $\Rarr$ $\exists y \in M: x - \e < y ≤ x$ Dabei $y^2 ≤ a$\\
$x^2 - a ≤ x^2 - y^2 = (x+y)(x-y) < 2x\e - x^2 -a $ \lightning
\\
Annahme 2:\\
$x^2 < a \Rarr \frac{a}{x^2} > 1 \Rarr \frac{a}{x^2} - 1 > 0$\\
Sei $r:= min(1, \frac{1}{3}\left(\frac{a}{x^2} - 1\right)) > 0$\\
$\Rarr (1+r)^2 = 1 + (2+r) \cdot r < 1 + 3r ≤ \frac{a}{x^2}$\\
$\Rarr [x(1+r)^2] ≤ a$\\
$\Rarr x(1+r) \in M$ aber : $x(1+r) > x$
}
\section{Rechenregel zu Wurzeln}
Seien $x, y \in \R, x, y> 0$  und $k\nN\Rarr\sqrt[k]{x}\cdot \sqrt[k]{y} = \sqrt[k]{x\cdot y}$\\
\subsubsection*{Beweis:}
$(\sqrt[k]{x} \cdot \sqrt[k]{x})^k = (\sqrt[k]{x})^k \cdot (\sqrt[k]{x})^k = x \cdot y \Rarr$ Behauptung.\\
Wir wissen: $\sqrt{2} \in \R \setminus \Q$. Also $\Q \subseteq \R$\\
Die Zahlen aus $\R \setminus \Q$ heißen irrational.
\bem
Im Körper $\Q$ ist das Vollständigkeitsaxiom \ul{nicht} erfüllt, das heißt, es gibt Mengen $M \subseteq \Q, M\neq \emptyset, M$ beschränkt, so dass $M$ kleine kleinste obere Schranke im $\Q$ hat.\\
Denn: sonst würde Satz 3.16 auch im $\Q$ gelten $\Rarr x^2 = 2$ halt eine Lösung in $\Q$ \lightning
\chapter{Funktionen}
\section{Definition}
Seien $X, Y$ Mengen, eine \ul{Abbildung} $f$ von $X$ nach $Y$ ist eine Vorschrift, die jedem $x \in X$ genau ein $y\in f(x) \in Y$ zuordnet.\\
Schreibweise: $f: X \to Y, x \mapsto f(x)$\\
\begin{description}
\item[X] Definitionsbereich von $f$
\item[Y] Zielbereich von $f$
\item[f(x)] Bild von $x$ unter $f$
\end{description}
\subsection*{Graph von $f$}
$\Gamma_f = \{x, \}$
% Kopfzeile beim Kapitelanfang:
\fancypagestyle{plain}{
%Kopfzeile links bzw. innen
\fancyhead[L]{\calligra\Large Vorlesung Nr. 6}
%Kopfzeile rechts bzw. außen
\fancyhead[R]{\calligra\Large 31.10.2013}
}
%Kopfzeile links bzw. innen
\fancyhead[L]{\calligra\Large Vorlesung Nr. 6}
%Kopfzeile rechts bzw. außen
\fancyhead[R]{\calligra\Large 31.10.2013}
% **************************************************
%
Abb. (Funktion) $f$ von $X$ nach $Y$ ($X, Y$ Mengen)\\
$f:X\to Y, x\mapsto f(x)$
\subsubsection*{Beispiel}
\enum{
\item $X = \{\text{Studenten an der UPB}\}, f:X\to \N, x\mapsto \text{Alle von }x$
\item $f:\R \to \R, x\to x^2$\\
$\Gamma_f:$ Normalparabel\\
$f(\R) = \{x\in \R: x≥0 \}$\\
\begin{tikzpicture}[domain=0:2, scale=0.5, prefix="plots/", smooth]
    \draw[very thin,color=gray] (-0.1,-0.1) grid (4.1,4.1);
    \draw[->] (-0.2,0) -- (4.2,0) node[right] {$x$};
    \draw[->] (0,-0.1) -- (0,4.2) node[right] {$y$};
    \draw[color=red] plot[id=v06g1] function{x*x} node[right] {};
\end{tikzpicture}
\item Lineare Funktionen\\
$f: \R \to \R, x \to a \cdot x + b$\\
$\Gamma_f:$ Gerade\\
$a:$ Steigung\\
$b$: Achsenabschnitt\\
%TODO Steigungsdreieck einzeichnen.
\begin{tikzpicture}[domain=0:4, scale=0.5, prefix="plots/", smooth]
    \draw[very thin,color=gray] (-0.1,-0.1) grid (4.1,4.1);
    \draw[->] (-0.2,0) -- (4.2,0) node[right] {$x$};
    \draw[->] (0,-0.1) -- (0,4.2) node[right] {$y$};
    \draw[color=red] plot[id=v06g2] function{ 0.7*x + 1} node[right] {};
\end{tikzpicture}
\item Polynomfunktionen
$p: \R \to \R, p(x) = a_nx^n + \cdots + a_1x + a_0$ mit $n\nN_0, a_0 ... a_n Koeffizienten$
\item Floor-Funktion (Gauß-Klammer)
$\lfloor . \rfloor: \R \to \R$\\
$\lfloor x \rfloor := \text{größtes }k\in \Z \text{mit }k ≤ x$\\
\begin{tikzpicture}[domain=0:4, scale=0.5, prefix="plots/", samples=250]
    \draw[very thin,color=gray] (-0.1,-0.1) grid (4.1,4.1);
    \draw[->] (-0.2,0) -- (4.2,0) node[right] {$x$};
    \draw[->] (0,-0.1) -- (0,4.2) node[right] {$y$};
    \draw[color=red] plot[id=v06g3] function{floor(x)} node[right] {};
\end{tikzpicture}
\item Identische Abbildung auf einer Menge $X$\\
$id_x: X\to X, x\to x$
\item $P:=$ Menge aller Sortierprogramme für endliche Listen\\
$L:= P\times \N \to \R$\\
$L(p, n) :=$ max. Laufzeit, die ein Programm $p$ zum sortieren einer Liste der Länge $n$ braucht.
}
Für die Charakterisierung einer Abbildung ist neben der Abbildungsvorschrift $f$, auch der Definitionsbereich $X$ wichtig.
\subsubsection*{Beispiel}
$f:\R \to \R, f(x) = |x|$
und $g:[-1, 1] \to \R, g(x) = |x|$ sind verschiedene Abbildungen. $g$ ist eine Restriktion von $f$.
\section{Definition}
Sei $f: X\to Y$ Abbildung
\enum {
\item Sei $A\subseteq X, f(A):= \{f(x) : x \in A\}$\\
$f(x) \subseteq Y:$ Wertebereich von $f$\\
\item Sei $B\subseteq Y: f^{-1}(B):=\{x\in X: f(x) \in B\}$\\
Urbild von $B$ unter $f$.\\
Besagt nicht, dass $f^{-1}$ Abbildung!
}
\subsubsection*{Beispiel}
$f:\Z \to \Z, x \mapsto x^2$\\
$f(\{-2, 5\}) = \{4, 25\}$\\
$f^{-1}(\{4, 25\}) = \{± 2, ±5\}$\\
$f^{-1}(\{3\}) = \emptyset$
\section{Definition: Komposition von Abbildungen}
Seien $f:X\to Y$ und $g: Y\to Z$ Abbildungen\\
Die Komposition (Verknüpfung, Verkettung) von $f$ und $g$ ist die Abbildung\\
$g\circ f: X\to Z, x \to g(f(x))$\\
%TODO Schaubild Transitive Verkettung
($g$ nach $f$)
\subsubsection*{Beispiel}
$f, g: \R \to \R, f(x) = 2x+1 , g(x) = x^2$\\
$(g \circ f)(x) = (2x+1)^2 = 4x^2 + 4x + 1$\\
$(f \circ g)(x) = 2x^2 + 1$\\
$\Rarr (g \circ f) \neq (f \circ g)$\\
Aber Komposition ist Assoziativ
\section{Satz}
Seien $f:X\to Y, g:Y\to Z, h:Z\to W$ Abbildungen $\Rarr$ $h \circ (g \circ f) = (h \circ g) \circ f$\\
Beweis: $(h\circ (g\circ f))(x) = h((g\circ f)(x) = h((g(f(x)))) = (h \circ g)(f(x)) = ((h \circ g) \circ f)(x) \qed$
\section{Eigenschaften von Abbildungen}
\begin{description}
\item[injektiv] falls es zu jedem $y \in Y$ höchstens ein $x \in X$ gibt mit $f(x) = y$
\item[surjektiv] falls es zu jedem $y \in Y$ mindestens ein $x \in X$ gibt mit $f(x) = y$
\item[bijektiv] falls es zu jedem $y \in Y$ genau ein $x \in X$ gibt mit $f(x) = y$\\
$f(x)$ ist surjektiv und bijektiv
\end{description}
Sei $f:X\to Y$ bijektiv $\forall y\in Y \exists! x\in X: f(x) = y$\\
Wir können dann $g: Y\to X$ definieren durch $g(y) := x$ falls $y = f(x)$\\
damit $g(f(x)) = x\ \forall x\in X, f(g(y)) = y \forall y\in Y$\\
Das heißt: $g\circ f = id_x$, $f \circ g = id_y$
\section{Definition: Umkehrabbildung}
$g$ heißt die Umkehrabbildung (Umkehrfunktion) von $f$. \desc{Bezeichnung:}{$g= f^{-1}$}
Damit falls $f$ bijektiv: $y = f(x) \Leftrightarrow x = f^{-1}(y)$
\section{Satz}
Für $f: X\to Y$ sind äquivalent:
\enum {
\item $f$ ist bijektiv
\item $\exists g: Y\to X: g\circ f = id_x, f \circ g = id_y$
 in diesem Fall ist $g = f^{-1}$
}
\subsubsection*{Beweis}
\begin{description}
\item[1 $\Rarr$ 2] siehe oben
\item[2 $\Rarr$ 1] $f$ injektiv, denn: sei $f(x_1) = f(x_2) \Rarr x_1 = g(f(x_1)) = g(f(x_2)) = x_2$\\
$f$ surjektiv, denn: Sei $y \in Y \Rarr y = f(g(y))$\\
$g$ in (2) ist eindeutig, da $g(f(x)) = x$ und $f$ surjektiv$\qed$
\end{description}
Sei $f:X\to Y$ bijektiv\\
$X, Y \subseteq \R$, Graph von $f^{-1}$?
$\Gamma_f = \{(x, f(x)): x \in X\}$\\
$\Gamma_{f^{-1}} = \{(f(x), x)\}$\\
entsteht aus $\Gamma_f$ durch Spiegelung an der Hauptdiagonalen $x = y$\\
\begin{tikzpicture}[domain=-2:2, scale=0.5, prefix="plots/", smooth]
    \draw[very thin,color=gray] (-2.1,-2.1) grid (3.1,3.1);
    \draw[->] (-2,0) -- (3.2,0) node[right] {$x$};
    \draw[->] (0,-2.1) -- (0,3.2) node[right] {$y$};
    \draw[color=red, domain=-2:3] plot[id=v06g4] function{log(x)} node[right] {$log(x)$};
    \draw[color=green, domain=-2:1.35] plot[id=v06g5] function{exp(x)} node[right] {$exp(x)$};
\end{tikzpicture}
\end{document}
